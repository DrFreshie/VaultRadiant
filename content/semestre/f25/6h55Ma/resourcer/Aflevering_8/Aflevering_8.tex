\documentclass[10pt,a4paper,table]{article}
	
	\usepackage{amsmath,amssymb,amsthm,mathtools,amsfonts}
	\usepackage{breqn}
	\usepackage[utf8]{inputenc}
	\usepackage[danish]{babel}
	\usepackage[T1]{fontenc}
	
	%%%%%%%%%%%%%%%%%%%%%%%%%%%%%%%%%%%%%%%%%%%%%%%%%%%%%%%%%%%%%%%%%%%%%%%%%%%%%% Fonts
	
	\usepackage{fontspec}
	\setmainfont{Times New Roman}
	\newfontfamily\garamond[Numbers=OldStyle]{EB Garamond}
	\newfontfamily\plantin[Numbers=OldStyle]{Plantin MT Pro}
	\newfontfamily\wal{Walbaum Com}
	\newfontfamily\klav{Klavika}
	\newfontfamily\klavl{Klavika light}
	\newfontfamily\quotefont[Ligatures=TeX]{Linux Libertine O}
	\newfontfamily\vest{TRY Vesterbro Regular}
	\newfontfamily\vestb{TRY Vesterbro Poster}
	\newfontfamily\overp{Overpass}
	
	%%%%%%%%%%%%%%%%%%%%%%%%%%%%%%%%%%%%%%%%%%%%%%%%%%%%%%%%%%%%%%%%%%%%%%%%%%%%%% Colors
	
	\usepackage{xcolor}
	
	\definecolor{frbblue}{RGB}{47, 88, 109}
	\definecolor{hgred}{RGB}{140, 10, 10}
	\definecolor{frbgreenl}{RGB}{162, 202, 137}
	\definecolor{frbgreen}{RGB}{28, 85, 66}
	\definecolor{frbgrey}{RGB}{243, 245, 242}
	
	%%%%%%%%%%%%%%%%%%%%%%%%%%%%%%%%%%%%%%%%%%%%%%%%%%%%%%%%%%%%%%%%%%%%%%%%%%%%%% Other Graphics
	
	\usepackage{graphicx}
	\graphicspath{{../img/}}
	\usepackage{tikz}
	\usepackage{placeins}
	\usepackage{subcaption}
	
	\usepackage[pdfborder={0 0 0}]{hyperref} %% Ingen firkant rundt om referencer

	\usepackage{multicol}
	\usepackage{cellspace}
		\setlength\cellspacetoplimit{100pt}
		\setlength\cellspacebottomlimit{100pt}
	\usepackage{tabularx}
		\newcolumntype{b}{>{\hsize=\hsize}>{\centering\arraybackslash}X}
	\usepackage{siunitx}
	\usepackage{cellspace}
		\setlength\cellspacetoplimit{4pt}
		\setlength\cellspacebottomlimit{4pt}
	
	\usepackage{enumitem}% better controls with enumerating
	\usepackage{wrapfig}
	\usepackage{caption}
	%\usepackage{dirtytalk}
	%\usepackage{tasks}
	\usepackage{titlesec}

	\setlength{\parskip}{0.5\baselineskip}
	\setlength{\parindent}{0cm}
	\setlist[enumerate]{label=\alph*),left=20pt}
	%\setlist[enumerate]{leftmargin=\parindent}
	%\setlist[enumerate]{nosep}
	
	
	%%%%%%%%%%%%%%%%%%%%%%%%%%%%%%%%%%%%%%%%%%%%%%%%%%%%%%%%%%%%%%%%%%%%%%%%%%%%%% New commands
	
	\newcommand\svar[1]{\newline\textcolor{red}{\textbf{Svar}\\#1}}
	
	\usepackage[h]{esvect}
	\newcommand{\twvect}[2]{\ensuremath{\begin{pmatrix}#1\\#2\end{pmatrix}}}
	
	\newcommand*\quotesize{20}
	\newcommand*{\openquote}
		{\tikz[remember picture,overlay,xshift=-3ex,yshift=-0ex]
			\node (OQ) {\quotefont\fontsize{\quotesize}{\quotesize}\selectfont``};\kern0pt}

	\newcommand*{\closequote}[1]
		{\tikz[remember picture,overlay,xshift=2ex,yshift={#1}]
			\node (CQ) {\quotefont\fontsize{\quotesize}{\quotesize}\selectfont''};}
			
	\newcommand{\qm}[1]{``#1''}
	\newcommand{\iquote}[1]{\begin{quote}\itshape \openquote#1\hfill\closequote{0ex} \end{quote}}
	
	%%%%%%%%%%%%%%%%%%%%%%%%%%%%%%%%%%%%%%%%%%%%%%%%%%%%%%%%%%%%%%%%%%%%%%%%%%%%%% Sections
	\usepackage{chngcntr}
	
	\renewcommand\thesection{\arabic{section}}

	\titleformat{\section}[block] 
  		{\flushright\overp\Large\color{frbgreen}}    % format for whole title
  		{}                           % no separate label (empty)
  		{0pt}                        % spacing between label and title
		{DELPRØVE \thesection}       % prepend "Delprøve <number>"
		[{\color{frbgreenl}\vspace{-1em}\hspace*{-0.165\linewidth}\rule{\dimexpr1.165\linewidth\relax}{0.4pt}\kern1mm}]
	
	\counterwithout{subsection}{section}
	\renewcommand\thesubsection{Opgave \arabic{subsection}}
	\titleformat{\subsection}[leftmargin]{\bfseries}{}{0pt}{\hspace{-5em}\thesubsection}
	
	%%%%%%%%%%%%%%%%%%%%%%%%%%%%%%%%%%%%%%%%%%%%%%%%%%%%%%%%%%%%%%%%%%%%%%%%%%%%%% Header & Footer
	
	\usepackage{bigfoot}
	\DeclareNewFootnote{a}
		\renewcommand{\thefootnotea}{\fnsymbol{footnotea}}
			\newcommand{\footnotes}[1]{\footnotea[2]{#1}} % here you can specify what symbol you want in footnotes
			\MakePerPage{footnotes}
	
		\usepackage{fancyhdr}
	\fancypagestyle{plain}{%
		\setlength{\headheight}{60pt}%
		\setlength{\headsep}{0pt}
		\fancyhf{}% No header/footer
		\renewcommand{\footrulewidth}{0pt}% No footer rule
		\renewcommand\headrule{
			\centerline{\begin{minipage}{1\textwidth}
				\color{frbblue}\hrule width \hsize \kern 1mm \hrule width \hsize height 2pt \vspace{4em}
			\end{minipage}}
		}
		\fancyhead[L]{\hspace{0em}\includegraphics[height=2.5em]{frbvuc_logo.png}}
		\fancyhead[R]{\overp\color{frbgreen}\footnotesize Efterår 2025}
		\fancyhead[C]{\overp\color{frbgreenl}\normalsize \overp Mat C-B\\\Large\color{frbgreen}\vestb Aflevering 8}
		\cfoot{\thepage}
	}
	\pagestyle{fancy}
	\fancyhf{}
	\renewcommand\headrule{
			\centerline{\begin{minipage}{\textwidth}
				\color{frbgreen}\hrule width \hsize \kern 1mm
			\end{minipage}}}
	\lhead{\overp\color{frbgreen}\footnotesize Mat C-B}
	\rhead{\overp\color{frbgreen}\footnotesize Efterår 2025}
	\cfoot{\thepage}
	
	%%%%%%%%%%%%%%%%%%%%%%%%%%%%%%%%%%%%%%%%%%%%%%%%%%%%%%%%%%%%%%%%%%%%%%%%%%%%%% Document

\begin{document}
\thispagestyle{plain}

\section{}
\subsection{}
En funktion $f$ er givet ved
$$
f(x) = x^2 - 2x .
$$

\begin{enumerate}
    \item Bestem $f(5)$.
    \item Bestem $f'(5)$.
\end{enumerate}

\subsection{}
En funktion $f$ er givet ved
$$
f(x)=6\cdot\sqrt{x}\, .
$$

\begin{enumerate}
    \item Bestem $f'(x)$.
\end{enumerate}

En anden funktion $g$ er givet ved
$$
g(x)=k\cdot\sqrt{x}, \quad \text{hvor } k \text{ er et tal}.
$$
Det oplyses, at
$$
g'(4)=3 .
$$

\begin{enumerate}
    \setcounter{enumi}{1}
    \item Bestem tallet $k$.
\end{enumerate}

\subsection{}
\begin{wrapfigure}[5]{r}{0.22\textwidth}
  \vspace{-18pt}
  \centering
  \includegraphics[width=0.32\textwidth]{tangent} % din figurfil
\end{wrapfigure}

Figuren viser grafen for en differentiabel funktion $f$ samt tangenten til grafen i punktet $P(1,4)$.

Det oplyses, at $f'(1)=0{,}5$.

\begin{enumerate}
  \item Bestem en ligning for tangenten til grafen for $f$ i punktet $P(1,4)$.
\end{enumerate}

\subsection{}
En funktion er givet ved
$$
f(x) = x^3 \cdot \ln(x).
$$

\begin{enumerate}
    \item Bestem $f(1)$.
    \item Bestem $f'(x)$.
\end{enumerate}

%\clearpage
\section{}
\subsection{}
En funktion $f$ er givet ved
\[
    f(x) = 5\cdot\ln(x) + 4x .
\]

\begin{enumerate}
    \item Bestem en ligning for tangenten til grafen for $f$ i punktet med første-koordinat $x=1$.
\end{enumerate}

\subsection{}
En funktion $V$ har forskriften
$$
V(x)=-2\cdot x^{3}+10\cdot x^{2}, \qquad 0<x<5.
$$

\begin{enumerate}
    \item Bestem maksimum for $V$ ved brug af $V'(x)$.
\end{enumerate}

\begin{figure}[h!]
\centering
\includegraphics[width=0.35\textwidth]{kasse}
\end{figure}
\FloatBarrier
Figuren viser en retvinklet kasse med sidelængder $x$ cm, $x$ cm og $(10-2x)$ cm.  
\begin{enumerate}   
	\item Argumentér for, at forskriften $V(x)$ beskriver kassens rumfang som funktion af $x$.
\end{enumerate}

\subsection{}
En funktion $f$ er givet ved
$$
f(x)=2\cdot \ln\left(0,5x^{2}-x+1\right)+3 .
$$

\begin{enumerate}
    \item Bestem $f'(x)$.
    \item Bestem en ligning for tangenten til grafen for $f$ i punktet $(2,\,f(2))$.
\end{enumerate}

\subsection{}
Funktionen $f$ er givet ved
$$
f(x) = x^2 + 4\sqrt{x}\, .
$$

\begin{enumerate}
    \item Bestem en ligning for tangenten til grafen for $f$ i punktet $(1,\,f(1))$.
\end{enumerate}

\end{document}

%%%%%%%%%%%%%%%%%%%%%%%%%%%%%%%%%%%%%%%%%%%%%%%%%%%%%%%%%%%%%%%%%%%%%%%%%%%%%% Skabeloner

% Sidestillet figur
\begin{wrapfigure}[8]{r}{0.2\textwidth}
\vspace{-18pt}
\includegraphics[width=0.3\textwidth]{ovn}
\end{wrapfigure}

% Midterstillet figur
\begin{figure}[h!]
\centering
\includegraphics[width=0.8\textwidth]{abc}
\end{figure}

% Førstillet figur
\
\begin{figure}[h!]
\centering
\vspace{-15pt}
\includegraphics[width=0.35\textwidth]{stud}
\end{figure}

% Tabel
\begin{figure}[h!]
\centering\renewcommand{\arraystretch}{1.5}
\begin{tabularx}{0.9\textwidth}{|l|b|b|b|b|b|b|}
	\hline \cellcolor{hggreen} Decimaltal & 7\% & -51\% & 13,7\% & 126\% & 456\% & 0,28\%\\\hline
	\cellcolor{hggreen} Procenttal &  &  &  &  &  & \\\hline
\end{tabularx}
\end{figure}

% Forklaringsopgaver
\begin{align*}
&&	&						&&\underset{\rule{0.8\linewidth}{0pt}}{\textbf{Forklaring:}}\\[1em]
&&	&\frac{(x+2)^2 - 4}{x} 		&&\underset{\rule{0.8\linewidth}{0.4pt}}{\text{Udtrykket skrives op.}}\\[2em]
&&	=&\ \frac{x^2 + 4 + 4x - 4}{x}	&&\underset{\rule{0.8\linewidth}{0.4pt}}{\text{}}\\[2em]
&&	=&\ \frac{x^2 + 4x}{x}			&&\underset{\rule{0.8\linewidth}{0.4pt}}{\text{}}\\[2em]
&&	=&\ x + 4					&&\underset{\rule{0.8\linewidth}{0.4pt}}{\text{}}\\[2em]
\end{align*}

% Sørg for at figur er placeret før et vidst sted
\FloatBarrier




