\documentclass[10pt,a4paper,table]{article}
	
	\usepackage{amsmath,amssymb,amsthm,mathtools,amsfonts}
	\usepackage{breqn}
	\usepackage[utf8]{inputenc}
	\usepackage[danish]{babel}
	\usepackage[T1]{fontenc}
	
	%%%%%%%%%%%%%%%%%%%%%%%%%%%%%%%%%%%%%%%%%%%%%%%%%%%%%%%%%%%%%%%%%%%%%%%%%%%%%% Fonts
	
	\usepackage{fontspec}
	\setmainfont{Times New Roman}
	\newfontfamily\garamond[Numbers=OldStyle]{EB Garamond}
	\newfontfamily\plantin[Numbers=OldStyle]{Plantin MT Pro}
	\newfontfamily\wal{Walbaum Com}
	\newfontfamily\klav{Klavika}
	\newfontfamily\klavl{Klavika light}
	\newfontfamily\quotefont[Ligatures=TeX]{Linux Libertine O}
	\newfontfamily\vest{TRY Vesterbro Regular}
	\newfontfamily\vestb{TRY Vesterbro Poster}
	\newfontfamily\overp{Overpass}
	
	%%%%%%%%%%%%%%%%%%%%%%%%%%%%%%%%%%%%%%%%%%%%%%%%%%%%%%%%%%%%%%%%%%%%%%%%%%%%%% Colors
	
	\usepackage{xcolor}
	\usepackage{transparent}
	
	\definecolor{frbblue}{RGB}{47, 88, 109}
	\definecolor{hgred}{RGB}{140, 10, 10}
	\definecolor{frbgreenl}{RGB}{162, 202, 137}
	\definecolor{frbgreen}{RGB}{28, 85, 66}
	\definecolor{frbgrey}{RGB}{243, 245, 242}
	
	%%%%%%%%%%%%%%%%%%%%%%%%%%%%%%%%%%%%%%%%%%%%%%%%%%%%%%%%%%%%%%%%%%%%%%%%%%%%%% Other Graphics
	
	\usepackage{graphicx}
	\graphicspath{{../img/}}
	\usepackage{tikz}
	\usetikzlibrary{calc}
	\usepackage{placeins}
	
	\usepackage[pdfborder={0 0 0}]{hyperref} %% Ingen firkant rundt om referencer

	\usepackage{multicol}
	\usepackage{cellspace}
		\setlength\cellspacetoplimit{100pt}
		\setlength\cellspacebottomlimit{100pt}
	\usepackage{tabularx}
		\newcolumntype{b}{>{\hsize=\hsize}>{\centering\arraybackslash}X}
	\usepackage{siunitx}
	\usepackage{cellspace}
		\setlength\cellspacetoplimit{4pt}
		\setlength\cellspacebottomlimit{4pt}
	
	\usepackage{enumitem}% better controls with enumerating
	\usepackage{wrapfig}
	\usepackage{caption}
	%\usepackage{dirtytalk}
	%\usepackage{tasks}
	\usepackage{titlesec}

	\setlength{\parskip}{0.5\baselineskip}
	\setlength{\parindent}{0cm}
	\setlist[enumerate]{label=\alph*),left=20pt}
	%\setlist[enumerate]{leftmargin=\parindent}
	%\setlist[enumerate]{nosep}
	
	
	%%%%%%%%%%%%%%%%%%%%%%%%%%%%%%%%%%%%%%%%%%%%%%%%%%%%%%%%%%%%%%%%%%%%%%%%%%%%%% New commands
	
	\newcommand\svar[1]{\newline\textcolor{red}{\textbf{Svar}\\#1}}
	
	\usepackage[h]{esvect}
	\newcommand{\twvect}[2]{\ensuremath{\begin{pmatrix}#1\\#2\end{pmatrix}}}
	
	\newcommand*\quotesize{20}
	\newcommand*{\openquote}
		{\tikz[remember picture,overlay,xshift=-3ex,yshift=-0ex]
			\node (OQ) {\quotefont\fontsize{\quotesize}{\quotesize}\selectfont``};\kern0pt}

	\newcommand*{\closequote}[1]
		{\tikz[remember picture,overlay,xshift=2ex,yshift={#1}]
			\node (CQ) {\quotefont\fontsize{\quotesize}{\quotesize}\selectfont''};}
			
	\newcommand{\qm}[1]{``#1''}
	\newcommand{\iquote}[1]{\begin{quote}\itshape \openquote#1\hfill\closequote{0ex} \end{quote}}
	
	%%%%%%%%%%%%%%%%%%%%%%%%%%%%%%%%%%%%%%%%%%%%%%%%%%%%%%%%%%%%%%%%%%%%%%%%%%%%%% Sections
	\usepackage{chngcntr}
	
	\renewcommand\thesection{\arabic{section}}

	\titleformat{\section}[block] 
  		{\flushright\overp\Large\color{frbgreen}}    % format for whole title
  		{}                           % no separate label (empty)
  		{0pt}                        % spacing between label and title
		{DELPRØVE \thesection}       % prepend "Delprøve <number>"
		[{\color{frbgreenl}\vspace{-1em}\hspace*{-0.165\linewidth}\rule{\dimexpr1.165\linewidth\relax}{0.4pt}\kern1mm}]
	
	\counterwithout{subsection}{section}
	\renewcommand\thesubsection{Øvelse \arabic{subsection}}
	\titleformat{\subsection}[leftmargin]{\bfseries}{}{0pt}{\hspace{-5em}\thesubsection}
	
	%%%%%%%%%%%%%%%%%%%%%%%%%%%%%%%%%%%%%%%%%%%%%%%%%%%%%%%%%%%%%%%%%%%%%%%%%%%%%% Header & Footer
	
	\usepackage{bigfoot}
	\DeclareNewFootnote{a}
		\renewcommand{\thefootnotea}{\fnsymbol{footnotea}}
			\newcommand{\footnotes}[1]{\footnotea[2]{#1}} % here you can specify what symbol you want in footnotes
			\MakePerPage{footnotes}
	
		\usepackage{fancyhdr}
	\fancypagestyle{plain}{%
		\setlength{\headheight}{60pt}%
		\setlength{\headsep}{0pt}
		\fancyhf{}% No header/footer
		\renewcommand{\footrulewidth}{0pt}% No footer rule
		\renewcommand\headrule{}% No header rule
		\fancyhead[L]{}
		\fancyhead[C]{}
		\fancyhead[R]{\overp\color{frbgreenl}\normalsize \overp Mat C-B\\\huge\color{frbgreen}\vestb Polynomier}
		\cfoot{\thepage}
	}
	\pagestyle{fancy}
	\fancyhf{}
	\renewcommand\headrule{}
	\lhead{}
	\rhead{}
	\cfoot{\thepage}
	
	%%%%%%%%%%%%%%%%%%%%%%%%%%%%%%%%%%%%%%%%%%%%%%%%%%%%%%%%%%%%%%%%%%%%%%%%%%%%%% Document

\begin{document}
\thispagestyle{plain}
\begin{tikzpicture}[remember picture,overlay]
    \node[anchor=north west,yshift=-48pt,xshift=120pt,opacity=0.2]%
        at (current page.north west)
        {\includegraphics[scale=0.25]{frbvuc_logo.png}};
\end{tikzpicture}
\begin{tikzpicture}[remember picture,overlay]
  % a light grey/green like in your screenshot
  \definecolor{bgline}{RGB}{205,214,208}

  % center for the circles (placed a bit *outside* the page to the left)
  \coordinate (C) at ($(current page.north)+(35mm,-5mm)$);

  % two thin concentric circles
  \draw[bgline, line width=0.6pt] (C) circle (65mm);

  % optional vertical hairlines near the page margins
  \draw[bgline, line width=0.4pt]
    ($(current page.north east)+(0,0mm)$) ++(-28mm,0)
    -- ($(current page.south east)+(0,10mm)$) ++(-28mm,0);
\end{tikzpicture}

\vspace{4em}

\subsection{}
Bestem eventuelle skæringspunkter mellem parablerne for følgende par af andengradspolynomier:

\[
    f_1(x) = 2x^2 \quad \text{og} \quad f_2(x) = -3x^2
\]

\[
    g_1(x) = -4x^2 + 4x - 3 \quad \text{og} \quad g_2(x) = -3x^2
\]

\[
    h_1(x) = x^2 + 2x - 1 \quad \text{og} \quad h_2(x) = 5x^2 + x + 2
\]

\subsection{}
Løs følgende ligninger:

\begin{enumerate}
    \item $3(x-4) + 2 = 2x + 7$
    \item $x^2 - 5x + 6 = 0$
\end{enumerate}

\subsection{}
Bestem koordinatsættet til toppunktet for parablen givet ved
\[
    f(x) = -2x^2 + 8x - 5.
\]

\subsection{}
Tegn grafen for funktionen $f$ givet ved
\[
f(x) =
\begin{cases}
- 3(x+2)^2 + 16, & -4.3 \leq x \leq 0 \\
- 3(x-2)^2 + 16, & 0 < x \leq 4.3
\end{cases}
\]

Grafen ligner logoet for en international fastfoodkæde. Hvilken?

\subsection{}
Katrine og Asta skyder med en vakuum-kanon, som de har bygget i skolens makerspace. De afprøver den ved at skyde ud over vandet. De optager affyringen med kamera og kan derved få nogle sammenhængende værdier for kanonkuglens bevægelse i $x$- og $y$-retningen. $x$ er den vandrette afstand fra det sted kanonen står, og $y$ er kuglens lodrette højde over vandoverfladen.

\begin{figure}[h!]
\centering\renewcommand{\arraystretch}{1.5}
\begin{tabularx}{\textwidth}{|l|b|b|b|b|b|b|b|b|b|}
    \hline \cellcolor{frbgrey} $x$ (m) & 0 & 1 & 2 & 3 & 5 & 7 & 9 & 12 & 15 \\\hline
    \cellcolor{frbgrey} $y$ (m) & 2 & 2,99 & 3,92 & 4,89 & 6,79 & 8,5 & 9,83 & 12,3 & 14,1 \\\hline
\end{tabularx}
\end{figure}

Ifølge den klassiske fysik kan en kugles bevægelse beskrives ved modellen
\[
    f(x)=a x^{2}+b x+c,
\]
hvor $x$ er den vandrette afstand, og $f(x)$ er kuglens højde.

\begin{enumerate}
    \item Bestem konstanterne $a$, $b$ og $c$ ved regression.
    \item Opskriv et udtryk for kuglens lodrette højde $f(x)$ som funktion af $x$.
    \item Tegn grafen for $f$.
    \item Bestem ud fra modellen, hvor langt kuglen har bevæget sig i vandret retning, når den rammer vandet.
    \item Bestem ud fra modellen, hvor højt kuglen maksimalt har været oppe.
\end{enumerate}

\end{document}

%%%%%%%%%%%%%%%%%%%%%%%%%%%%%%%%%%%%%%%%%%%%%%%%%%%%%%%%%%%%%%%%%%%%%%%%%%%%%% Skabeloner

% Sidestillet figur
\begin{wrapfigure}[8]{r}{0.2\textwidth}
\vspace{-18pt}
\includegraphics[width=0.3\textwidth]{ovn}
\end{wrapfigure}

% Midterstillet figur
\begin{figure}[h!]
\centering
\includegraphics[width=0.8\textwidth]{abc}
\end{figure}

% Førstillet figur
\
\begin{figure}[h!]
\centering
\vspace{-15pt}
\includegraphics[width=0.35\textwidth]{stud}
\end{figure}

% Tabel
\begin{figure}[h!]
\centering\renewcommand{\arraystretch}{1.5}
\begin{tabularx}{0.9\textwidth}{|l|b|b|b|b|b|b|}
	\hline \cellcolor{hggreen} Decimaltal & 7\% & -51\% & 13,7\% & 126\% & 456\% & 0,28\%\\\hline
	\cellcolor{hggreen} Procenttal &  &  &  &  &  & \\\hline
\end{tabularx}
\end{figure}

% Forklaringsopgaver
\begin{align*}
&&	&						&&\underset{\rule{0.8\linewidth}{0pt}}{\textbf{Forklaring:}}\\[1em]
&&	&\frac{(x+2)^2 - 4}{x} 		&&\underset{\rule{0.8\linewidth}{0.4pt}}{\text{Udtrykket skrives op.}}\\[2em]
&&	=&\ \frac{x^2 + 4 + 4x - 4}{x}	&&\underset{\rule{0.8\linewidth}{0.4pt}}{\text{}}\\[2em]
&&	=&\ \frac{x^2 + 4x}{x}			&&\underset{\rule{0.8\linewidth}{0.4pt}}{\text{}}\\[2em]
&&	=&\ x + 4					&&\underset{\rule{0.8\linewidth}{0.4pt}}{\text{}}\\[2em]
\end{align*}

% Sørg for at figur er placeret før et vidst sted
\FloatBarrier




