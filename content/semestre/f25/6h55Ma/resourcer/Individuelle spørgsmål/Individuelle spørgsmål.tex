 %! TeX program = lualatex
\documentclass[10pt,a4paper]{article}
	
	\usepackage{amsmath,amssymb,amsthm,mathtools,amsfonts}
	\usepackage{breqn}
	\usepackage[utf8]{inputenc}
	\usepackage[danish]{babel}
	\usepackage[T1]{fontenc}
	
	%%%%%%%%%%%%%%%%%%%%%%%%%%%%%%%%%%%%%%%%%%%%%%%%%%%%%%%%%%%%%%%%%%%%%%%%%%%%%% Fonts
	
	\usepackage{fontspec}
	\setmainfont{EB Garamond}
	\newfontfamily\garamond[Numbers=OldStyle]{EB Garamond}
	\newfontfamily\plantin[Numbers=OldStyle]{Plantin MT Pro}
	\newfontfamily\wal{Walbaum Com}
	\newfontfamily\klav{Klavika}
	\newfontfamily\klavl{Klavika light}
	\newfontfamily\quotefont[Ligatures=TeX]{Linux Libertine O}
	\newfontfamily\vest{TRY Vesterbro Regular}
	\newfontfamily\vestb{TRY Vesterbro Poster}
	\newfontfamily\overp{Overpass}
	
	%%%%%%%%%%%%%%%%%%%%%%%%%%%%%%%%%%%%%%%%%%%%%%%%%%%%%%%%%%%%%%%%%%%%%%%%%%%%%% Colors
	
	\usepackage{xcolor}
	
	\definecolor{frbblue}{RGB}{47, 88, 109}
	\definecolor{frbred}{RGB}{159, 37, 37}
	\definecolor{frbgreenl}{RGB}{162, 202, 137}
	\definecolor{frbgreen}{RGB}{28, 85, 66}
	\definecolor{frbgrey}{RGB}{243, 245, 242}
	
	%%%%%%%%%%%%%%%%%%%%%%%%%%%%%%%%%%%%%%%%%%%%%%%%%%%%%%%%%%%%%%%%%%%%%%%%%%%%%% Other Graphics
	
	\usepackage{graphicx}
	\graphicspath{{../img/}}
	\usepackage{tikz}
	\usepackage{placeins}
  \usepackage{fontawesome5}
	
	\usepackage[pdfborder={0 0 0}]{hyperref} %% Ingen firkant rundt om referencer

	\usepackage{multicol}
	\usepackage{cellspace}
		\setlength\cellspacetoplimit{100pt}
		\setlength\cellspacebottomlimit{100pt}
	\usepackage{tabularx}
		\newcolumntype{b}{>{\hsize=\hsize}>{\centering\arraybackslash}X}
	\usepackage{siunitx}
	\usepackage{cellspace}
		\setlength\cellspacetoplimit{4pt}
		\setlength\cellspacebottomlimit{4pt}
	
	\usepackage{enumitem}% better controls with enumerating
	\usepackage{wrapfig}
	\usepackage{caption}
	%\usepackage{dirtytalk}
	%\usepackage{tasks}
	\usepackage{titlesec}

	\setlength{\parskip}{0.5\baselineskip}
	\setlength{\parindent}{0cm}
	\setlist[enumerate]{}
	%\setlist[enumerate]{leftmargin=\parindent}
	%\setlist[enumerate]{nosep}
	
	
	%%%%%%%%%%%%%%%%%%%%%%%%%%%%%%%%%%%%%%%%%%%%%%%%%%%%%%%%%%%%%%%%%%%%%%%%%%%%%% New commands
	
	\newcommand\svar[1]{\newline\textcolor{red}{\textbf{Svar}\\#1}}
	
	\usepackage[h]{esvect}
	\newcommand{\twvect}[2]{\ensuremath{\begin{pmatrix}#1\\#2\end{pmatrix}}}
	
	\newcommand*\quotesize{20}
	\newcommand*{\openquote}
		{\tikz[remember picture,overlay,xshift=-3ex,yshift=-0ex]
			\node (OQ) {\quotefont\fontsize{\quotesize}{\quotesize}\selectfont``};\kern0pt}

	\newcommand*{\closequote}[1]
		{\tikz[remember picture,overlay,xshift=2ex,yshift={#1}]
			\node (CQ) {\quotefont\fontsize{\quotesize}{\quotesize}\selectfont''};}
			
	\newcommand{\qm}[1]{``#1''}
	\newcommand{\iquote}[1]{\begin{quote}\itshape \openquote#1\hfill\closequote{0ex} \end{quote}}
	
	\newcommand{\hint}[3][]{%
  		\marginpar{%
   			 \vspace{-#3pt}%	
			 \faBook \ \klavl\scriptsize#2%
			 \ifx\relax#1\relax\else\\[5pt]\faYoutube \ \klavl\scriptsize\color{frbblue}\href{#1}{Videobevis}\fi%
			 	}%
			}

	
	%%%%%%%%%%%%%%%%%%%%%%%%%%%%%%%%%%%%%%%%%%%%%%%%%%%%%%%%%%%%%%%%%%%%%%%%%%%%%% Sections
	
	%\setcounter{secnumdepth}{0}
	\usepackage{remreset} % continued subsection numbering
		\makeatletter
			\@removefromreset{subsection}{section}\renewcommand\thesubsection{\arabic{subsection}}
		\makeatother
	\renewcommand\thesection{\klavl\huge Del \arabic{section}}
	\titleformat{\section}[display]{\centering\overp}{\thesection}{-10pt}{\scshape\footnotesize}[\vspace{20pt}]
	\renewcommand\thesubsection{}
	\titleformat{\subsection}{\color{frbred}\large\overp}{}{}{}
	
	%%%%%%%%%%%%%%%%%%%%%%%%%%%%%%s%%%%%%%%%%%%%%%%%%%%%%%%%%%%%%%%%%%%%%%%%%%%%%%% Header & Footer
	
	\usepackage{bigfoot}
	\DeclareNewFootnote{a}
		\renewcommand{\thefootnotea}{\fnsymbol{footnotea}}
			\newcommand{\footnotes}[1]{\footnotea[2]{#1}} % here you can specify what symbol you want in footnotes
			\MakePerPage{footnotes}
	
		\usepackage{fancyhdr}
	\fancypagestyle{plain}{%
		\setlength{\headheight}{60pt}%
		\setlength{\headsep}{0pt}
		\fancyhf{}% No header/footer
		\renewcommand{\footrulewidth}{0pt}% No footer rule
		\renewcommand\headrule{
			\centerline{\begin{minipage}{1\textwidth}
				\color{frbblue}\hrule width \hsize \kern 1mm \hrule width \hsize height 2pt 
			\end{minipage}}
		}
		\fancyhead[L]{\hspace{0em}\includegraphics[height=2.5em]{frbvuc_logo.png}}
		\fancyhead[R]{\overp\color{frbgreen}\footnotesize Efterår 2025}
		\fancyhead[C]{\overp\color{frbblue}\normalsize 6H55 Ma: mundtlig eksamen\\\Large\vestb\color{frbgreen} Individuelle spørgsmål}
		\cfoot{\thepage}
	}
	\pagestyle{fancy}
	\fancyhf{}
	\renewcommand\headrule{
			\centerline{\begin{minipage}{\textwidth}
				\color{frbblue}\hrule width \hsize \kern 1mm
			\end{minipage}}}
	\lhead{\klavl\color{frbgreen}\footnotesize 6H55 2508 Ma}
	\rhead{\klavl\color{frbgreen}\footnotesize Efterår 2025}
	\cfoot{\thepage}
	
	%%%%%%%%%%%%%%%%%%%%%%%%%%%%%%%%%%%%%%%%%%%%%%%%%%%%%%%%%%%%%%%%%%%%%%%%%%%%%% Document

\begin{document}

\thispagestyle{plain}
\

\subsection*{Andengradspolynomier}
\noindent
\begin{enumerate}
  \item Gør,\marginpar{\href{https://o365itcfyn-my.sharepoint.com/:b:/g/personal/art_frbvuc_dk/IQDRWcKjWS1rQ6hwW-sGPBWxAbnUHmcCgFnwIDXYUZQg62k?e=32jXMg}{\overp\color{frbblue}\faFile\ Vejledning}} for et andengradspolynomium $f(x)=ax^2+bx+c$, rede for formlen til udregning af polynomiets rødder:
  $$x=\frac{-b\pm \sqrt{b^2-4ac}}{2a},$$
og forklar hvordan grafen for andengradspolynomiet afhænger af $a$ og diskriminanten $d$.
\item Gør,\marginpar{\href{https://o365itcfyn-my.sharepoint.com/:b:/g/personal/art_frbvuc_dk/IQCjvSBESnymTorfKMhkLsOHAWhxwlPZegIo6aKts-a4d0Q?e=4CqbN}{\overp\color{frbblue}\faFile\ Vejledning}} for et andengradspolynomium $f(x)=ax^2+bx+c$, rede for formlen til udregning af polynomiets toppunkt:
$$T=\left(\frac{-b}{2a},\frac{-d}{4a}\right),$$
og forklar hvordan grafen for andengradspolynomiet afhænger af $a$, $b$, og $c$.
\end{enumerate}

\subsection*{Geometri}
\begin{enumerate}[resume]
\item Gør,\marginpar{\href{https://o365itcfyn-my.sharepoint.com/:b:/g/personal/art_frbvuc_dk/IQAkmYK066VfSKFW5hUWetGoARolvZXpg04is2YxBlTs_HA?e=uYq1Dv}{\overp\color{frbblue}\faFile\ Vejledning}} for to ortogonale linjer $y=ax+b$ og $y=cx+d$, rede for formlen for produktet af deres hældninger:
$$a\cdot c=-1,$$
og forklar med et eksempel, hvordan man kan finde et skæringspunkt mellem to linjer.
\item Gør,\marginpar{\href{https://o365itcfyn-my.sharepoint.com/:b:/r/personal/art_frbvuc_dk/Documents/Matematik/Beviser/Bevis_49.pdf?csf=1&web=1&e=apdJFt}{\overp\color{frbblue}\faFile\ Vejledning}} for to punkter i et koordinatsystem, rede for formlen for afstanden mellem dem:
$$|AB| = \sqrt{(x_2-x_1)^2+(y_2-y_1)^2}.$$
Gør\marginpar{\href{https://o365itcfyn-my.sharepoint.com/:b:/g/personal/art_frbvuc_dk/IQCfYaEBxPpdTpfDnPdv_R4-Af2nsMHNJifpl-SXce6omxQ?e=JRreSp}{\overp\color{frbblue}\faFile\ Vejledning}} desuden, for en cirkel i et koordinatsystem med centrum i $C(a,b)$ og radius $r$, rede for cirklens ligning
$$(x-a)^2+(y-b)^2=r^2,$$
og forklar, hvordan man kan finde en ligning for tangenten til en cirkel.
\end{enumerate}

\subsection*{Funktioner}
\begin{enumerate}[resume]
\item Gør\marginpar{\href{https://o365itcfyn-my.sharepoint.com/:b:/g/personal/art_frbvuc_dk/IQAJZlvzMCfRSI3fkWkfq9LuAXgyWIuyN2E_4m_Hsz66PC8?e=6ZUs77}{\overp\color{frbblue}\faFile\ Vejledning}} rede for definitionen af logaritme funktionerne $\log(x)$ og $\ln(x)$, og udled logaritme-regnereglen:
$$\log(ab)=\log(a)+\log(b).$$
Forklar desuden funktionernes definitions- og værdimængde, samt hvordan man kan udregne deres monotoniforhold.
\item Gør\marginpar{\href{https://o365itcfyn-my.sharepoint.com/:b:/g/personal/art_frbvuc_dk/IQBuIwI1c05lQKnqL6ENCu_QAeGEixVPO3dHhv4RSdt8E2M?e=lynohg}{\overp\color{frbblue}\faFile\ Vejledning}} rede for definitionen af logaritme funktionerne $\log(x)$ og $\ln(x)$, og udled logaritme-regnereglen:
$$\log(a^x)=x\cdot \log(a).$$
Forklar desuden funktionernes definitions- og værdimængde, samt hvordan man kan udregne deres monotoniforhold.
\end{enumerate}

\subsection*{Differentialregning}
\begin{enumerate}[resume]
\item Gør,\marginpar{\href{https://o365itcfyn-my.sharepoint.com/:b:/g/personal/art_frbvuc_dk/IQC0qePjWk79Q40aSrTwf-wYAesPSA37M9DUe8rSxlaZu1g?e=26cgE0}{\overp\color{frbblue}\faFile\ Vejledning}} for to funktioner $f(x)$ og $g(x)$, rede for formlen for den afledte af deres sum:
$$\left(f(x)+g(x)\right)'=f'(x)+g'(x),$$
og forklar hvordan man kan differentiere deres produkt.
\item Gør,\marginpar{\href{https://o365itcfyn-my.sharepoint.com/:b:/g/personal/art_frbvuc_dk/IQAsywrJlTrCQ7f9tCQQ4KHiAb3W3KQB6iFKpY3cl646jeE?e=0DPRwq}{\overp\color{frbblue}\faFile\ Vejledning}} for funktionen $f(x)=x^2$, rede for formlen for dennes afledte:
$$f'(x)=2x,$$
og forklar hvordan man ved hjælp af differentialregning kan udregne en funktions monotoniforhold.
\item Gør,\marginpar{\href{https://o365itcfyn-my.sharepoint.com/:b:/g/personal/art_frbvuc_dk/IQAC8mBw9Qh2RbXlky6NCXWeAWoz4zTtSdcC-3k6MUKFGto?e=J99dW5}{\overp\color{frbblue}\faFile\ Vejledning}} for funktionen $f(x)=\sqrt{x}$, rede for formlen for dennes afledte:
$$f'(x)=\frac{1}{2\sqrt{x}},$$
og forklar hvad en sammensat funktion er, og hvordan man kan differentiere en sådan.
\item Gør,\marginpar{\href{https://o365itcfyn-my.sharepoint.com/:b:/g/personal/art_frbvuc_dk/IQBFj2WhWKeUQbO2QLd8uyDAAdnvsjTvXtgQfFUp_a3zg0E?e=s1jdhz}{\overp\color{frbblue}\faFile\ Vejledning}} for funktionen $f(x)=\frac{1}{x}$, rede for formlen for dennes afledte:
$$f'(x)=-\frac{1}{x^2},$$
og forklar hvordan differentialregning kan bruges til løse et optimeringsproblem.
\end{enumerate}

\subsection*{Sandsynlighedsregning}
\begin{enumerate}[resume]
\item Gør,\marginpar{\href{https://o365itcfyn-my.sharepoint.com/:b:/g/personal/art_frbvuc_dk/IQDayIov9mjkTobsSzbl4egXAS57DUSAOFh51LhN5pr3VJg?e=JmGdWP}{\overp\color{frbblue}\faFile\ Vejledning}} for en binomialfordelt stokastisk variabel $X$, rede for formlen til udregning af sandsynligheden:
$$P(X=r)=K(n,r)\cdot p^r\cdot (1-p)^{n-r},$$
og forklar hvordan man med en såkaldt binomialtest kan afprøve en statistisk hypotese.
\item Gør, for en binomialfordelt stokastisk variabel $X$, rede for formlen til udregning af sandsynligheden:
$$P(X=r)=K(n,r)\cdot p^r\cdot (1-p)^{n-r},$$
og forklar, med et eksempel, hvordan man med et såkaldt konfidensinterval kan undersøge størrelser i hele populationer ud fra stikprøver.
\item Gør, for en binomialfordelt stokastisk variabel $X$, rede for formlen til udregning af sandsynligheden:
$$P(X=r)=K(n,r)\cdot p^r\cdot (1-p)^{n-r},$$
og forklar, med et eksempel, hvordan deskriptorerne middelværdi og spredning kan bruges til at vurdere om et udfald er normalt eller exceptionelt.
\end{enumerate}












\end{document}


%%%%%%%%%%%%%%%%%%%%%%%%%%%%%%%%%%%%%%%%%%%%%%%%%%%%%%%%%%%%%%%%%%%%%%%%%%%%%% Ubrugte spørgsmål

\item Definer lineær vækst, eksponentiel vækst og potensvækst. Bevis for hver af de tre væksttyper hvad der sker med $y$-værdien når:
\begin{description}%[\alpha)]
\item[Lineær vækst:] $x$-værdien øges med 1
\item[Eksponentiel vækst:] $x$-værdien øges med 1
\item[Potensvækst:] $x$-værdien ganges med $k$
\end{description}

\item Gør rede for definitionen af cosinus, sinus, og tangens, og udled på baggrund heraf formlen for hældningsvinklen for en ret linje $f(x)=ax+b$:
$$\tan(v)=a,$$
og forklar med et eksempel, hvordan man kan finde vinklen mellem to linjer.
%%%%%%%%%%%%%%%%%%%%%%%%%%%%%%%%%%%%%%%%%%%%%%%%%%%%%%%%%%%%%%%%%%%%%%%%%%%%%% Skabeloner

% Sidestillet figur
\begin{wrapfigure}[8]{r}{0.2\textwidth}
\vspace{-18pt}
\includegraphics[width=0.3\textwidth]{ovn}
\end{wrapfigure}

% Midterstillet figur
\begin{figure}[h!]
\centering
\includegraphics[width=0.8\textwidth]{abc}
\end{figure}

% Førstillet figur
\
\begin{figure}[h!]
\centering
\vspace{-15pt}
\includegraphics[width=0.35\textwidth]{stud}
\end{figure}

% Tabel
\begin{figure}[h!]
\centering\renewcommand{\arraystretch}{1.5}
\begin{tabularx}{0.9\textwidth}{|l|b|b|b|b|b|b|}
	\hline \cellcolor{frbgreenl} Decimaltal & 7\% & -51\% & 13,7\% & 126\% & 456\% & 0,28\%\\\hline
	\cellcolor{frbgreenl} Procenttal &  &  &  &  &  & \\\hline
\end{tabularx}
\end{figure}

% Sørg for at figur er placeret før et vidst sted
\FloatBarrier


