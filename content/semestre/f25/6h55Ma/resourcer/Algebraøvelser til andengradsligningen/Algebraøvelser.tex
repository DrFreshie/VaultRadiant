\documentclass[a4paper,11pt]{article}
\usepackage[utf8]{inputenc}
\usepackage[T1]{fontenc}
\usepackage[danish]{babel}
\usepackage{amsmath}
\usepackage{enumitem}

\begin{document}

\section*{Opvarmningsopgaver: Gang ind i parentesen}
\begin{enumerate}
\item Omskriv udtrykkene ved at gange ind i parentesen.
\begin{enumerate}[label=\alph*)]

\item \(3(x+2)\)

\item \(-2(x-5)\)

\item \(4(2x+7)\)

\item \(-5(3x-4)\)

\item \(6(a+2b)\)

\item \(z(2x-5y)\)

\item \(a(x+3y+4)\)

\item \(-k(2a-b+3)\)

\item \(2m(n+p)\)

\item \(6a(a-b+c)\)

\item \(4a(ax^2+bx+c)\)

\end{enumerate}
\end{enumerate}

\clearpage

\section*{Opvarmningsopgaver: Fuldstændiggør kvadratet}

\begin{enumerate}
\item Omskriv udtrykkene så de står som kvadratet på en toleddet størrelse.

\begin{enumerate}[label=\alph*)]

\item \(x^2 + 4x + 4\)

\item \(x^2 - 6x + 9\)

\item \(x^2 + 10x + 25\)

\item \(x^2 - 8x + 16\)

\item \(x^2 + 12x + 36\)

\end{enumerate}

\item Omskriv ligningerne så venstre side står som kvadratet på en toleddet størrelse

\begin{enumerate}[label=\alph*)]

\item \(y^2 + 14y = -13\)

\item \(a^2 - 2a = 15\)

\item \(m^2 + 18m = 19\)

\item \(x^2 + 2x = 62\)

\item \(k^2 - 20k = -75\)

\end{enumerate}
\end{enumerate}
\end{document}