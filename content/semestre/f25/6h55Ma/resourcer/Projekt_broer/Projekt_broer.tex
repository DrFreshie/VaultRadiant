\documentclass[10pt,a4paper,table]{article}
	
	\usepackage{amsmath,amssymb,amsthm,mathtools,amsfonts}
	\usepackage{breqn}
	\usepackage[utf8]{inputenc}
	\usepackage[danish]{babel}
	\usepackage[T1]{fontenc}
	
	%%%%%%%%%%%%%%%%%%%%%%%%%%%%%%%%%%%%%%%%%%%%%%%%%%%%%%%%%%%%%%%%%%%%%%%%%%%%%% Fonts
	
	\usepackage{fontspec}
	\setmainfont{Times New Roman}
	\newfontfamily\garamond[Numbers=OldStyle]{EB Garamond}
	\newfontfamily\plantin[Numbers=OldStyle]{Plantin MT Pro}
	\newfontfamily\wal{Walbaum Com}
	\newfontfamily\klav{Klavika}
	\newfontfamily\klavl{Klavika light}
	\newfontfamily\quotefont[Ligatures=TeX]{Linux Libertine O}
	\newfontfamily\vest{TRY Vesterbro Regular}
	\newfontfamily\vestb{TRY Vesterbro Poster}
	\newfontfamily\overp{Overpass}
	
	%%%%%%%%%%%%%%%%%%%%%%%%%%%%%%%%%%%%%%%%%%%%%%%%%%%%%%%%%%%%%%%%%%%%%%%%%%%%%% Colors
	
	\usepackage{xcolor}
	\usepackage{transparent}
	
	\definecolor{frbblue}{RGB}{47, 88, 109}
	\definecolor{hgred}{RGB}{140, 10, 10}
	\definecolor{frbgreenl}{RGB}{162, 202, 137}
	\definecolor{frbgreen}{RGB}{28, 85, 66}
	\definecolor{frbgrey}{RGB}{243, 245, 242}
	
	%%%%%%%%%%%%%%%%%%%%%%%%%%%%%%%%%%%%%%%%%%%%%%%%%%%%%%%%%%%%%%%%%%%%%%%%%%%%%% Other Graphics
	
	\usepackage{graphicx}
	\graphicspath{{../img/}}
	\usepackage{tikz}
	\usetikzlibrary{calc}
	\usepackage{placeins}
	
	\usepackage[pdfborder={0 0 0}]{hyperref} %% Ingen firkant rundt om referencer

	\usepackage{multicol}
	\usepackage{cellspace}
		\setlength\cellspacetoplimit{100pt}
		\setlength\cellspacebottomlimit{100pt}
	\usepackage{tabularx}
		\newcolumntype{b}{>{\hsize=\hsize}>{\centering\arraybackslash}X}
	\usepackage{siunitx}
	\usepackage{cellspace}
		\setlength\cellspacetoplimit{4pt}
		\setlength\cellspacebottomlimit{4pt}
	
	\usepackage{enumitem}% better controls with enumerating
	\usepackage{wrapfig}
	\usepackage{caption}
	%\usepackage{dirtytalk}
	%\usepackage{tasks}
	\usepackage{titlesec}

	\setlength{\parskip}{0.5\baselineskip}
	\setlength{\parindent}{0cm}
	\setlist[enumerate]{label=\alph*),left=20pt}
	%\setlist[enumerate]{leftmargin=\parindent}
	%\setlist[enumerate]{nosep}
	
	
	%%%%%%%%%%%%%%%%%%%%%%%%%%%%%%%%%%%%%%%%%%%%%%%%%%%%%%%%%%%%%%%%%%%%%%%%%%%%%% New commands
	
	\newcommand\svar[1]{\newline\textcolor{red}{\textbf{Svar}\\#1}}
	
	\usepackage[h]{esvect}
	\newcommand{\twvect}[2]{\ensuremath{\begin{pmatrix}#1\\#2\end{pmatrix}}}
	
	\newcommand*\quotesize{20}
	\newcommand*{\openquote}
		{\tikz[remember picture,overlay,xshift=-3ex,yshift=-0ex]
			\node (OQ) {\quotefont\fontsize{\quotesize}{\quotesize}\selectfont``};\kern0pt}

	\newcommand*{\closequote}[1]
		{\tikz[remember picture,overlay,xshift=2ex,yshift={#1}]
			\node (CQ) {\quotefont\fontsize{\quotesize}{\quotesize}\selectfont''};}
			
	\newcommand{\qm}[1]{``#1''}
	\newcommand{\iquote}[1]{\begin{quote}\itshape \openquote#1\hfill\closequote{0ex} \end{quote}}
	
	%%%%%%%%%%%%%%%%%%%%%%%%%%%%%%%%%%%%%%%%%%%%%%%%%%%%%%%%%%%%%%%%%%%%%%%%%%%%%% Sections
	\usepackage{chngcntr}
	
	\renewcommand\thesection{\arabic{section}}

	\titleformat{\section}[block] 
  		{\flushleft\overp\Large\color{frbgreen}}    % format for whole title
  		{}                           % no separate label (empty)
  		{0pt}                        % spacing between label and title
		{\hspace{-6em}\begin{minipage}[t]{0.2\textwidth}\raggedleft DEL \thesection\\\footnotesize\vest}       % prepend "Delprøve <number>"
		[\end{minipage}\vspace{-2.9em}]
	
	\counterwithout{subsection}{section}
	\renewcommand\thesubsection{Øvelse \arabic{subsection}}
	\titleformat{\subsection}[leftmargin]{\bfseries}{}{0pt}{\hspace{-5em}\thesubsection}
	
	%%%%%%%%%%%%%%%%%%%%%%%%%%%%%%%%%%%%%%%%%%%%%%%%%%%%%%%%%%%%%%%%%%%%%%%%%%%%%% Header & Footer
	
	\usepackage{bigfoot}
	\DeclareNewFootnote{a}
		\renewcommand{\thefootnotea}{\fnsymbol{footnotea}}
			\newcommand{\footnotes}[1]{\footnotea[2]{#1}} % here you can specify what symbol you want in footnotes
			\MakePerPage{footnotes}
	
		\usepackage{fancyhdr}
	\fancypagestyle{plain}{%
		\setlength{\headheight}{60pt}%
		\setlength{\headsep}{0pt}
		\fancyhf{}% No header/footer
		\renewcommand{\footrulewidth}{0pt}% No footer rule
		\renewcommand\headrule{
			\centerline{\begin{minipage}{1\textwidth}
				\color{frbblue}\hrule width \hsize \kern 1mm
			\end{minipage}}
		}
		\fancyhead[L]{\hspace{0em}\includegraphics[height=2.5em]{frbvuc_logo.png}}
		\fancyhead[R]{\overp\color{frbgreen}\footnotesize Efterår 2025}
		\fancyhead[C]{\overp\color{frbgreenl}\normalsize \overp Mat C-B\\\Large\color{frbgreen}\vestb Projekt\\\normalsize\overp Cirkler og polynomier}
		\cfoot{\thepage}
	}
	\pagestyle{fancy}
	\fancyhf{}
	\renewcommand\headrule{
			\centerline{\begin{minipage}{\textwidth}
				\color{frbgreen}\hrule width \hsize \kern 1mm
			\end{minipage}}}
	\lhead{\overp\color{frbgreen}\footnotesize Mat C-B}
	\rhead{\overp\color{frbgreen}\footnotesize Efterår 2025}
	\cfoot{\thepage}
	
	%%%%%%%%%%%%%%%%%%%%%%%%%%%%%%%%%%%%%%%%%%%%%%%%%%%%%%%%%%%%%%%%%%%%%%%%%%%%%% Document

\begin{document}
\thispagestyle{plain}
\begin{tikzpicture}[remember picture,overlay]
  % a light grey/green like in your screenshot
  \definecolor{bgline}{RGB}{205,214,208}

  % center for the circles (placed a bit *outside* the page to the left)
  \coordinate (C) at ($(current page.north)+(25mm,35mm)$);

  % two thin concentric circles
  \draw[frbblue, line width=0.4pt, opacity=0.3] (C) circle (95mm);
\end{tikzpicture}

\vspace{1em}


\section{Badebro i Århus}
\begin{figure}[h!]
\centering
\includegraphics[width=0.8\textwidth]{bro1}
\end{figure}
\qm{Uendelighedsbroen} i Århusbugten er starten på et større kunstværk. Kunstværket skal til slut bestå af fem forbundne cirkelformede broer.

\begin{wrapfigure}[6]{r}{0.3\textwidth}
\vspace{-18pt}
\includegraphics[width=0.5\textwidth]{bro2}
\end{wrapfigure}
Den endelige udformning af kunstværket ses på skitsen til højre, hvor kunstværket er indlagt i et koordinatsystem. Vi kigger blot på den yderste rand af broerne i modellen og den midterste cirkel er indlagt med centrum i $(0,0)$ i koordinatsystemet.

Det oplyses, at den midterste bro (den der ses på øverste billede) har en diameter på 60 m. 
\begin{enumerate}
\item Opskriv en ligning for den cirkel $C_1$ (uden at der er skåret i den – altså den på øverste billede), der beskriver den midterste badebro.
\item Hvor lang er broen (ydre rand)?
\end{enumerate}

Broen er 2m bred.
\begin{enumerate}[resume]
\item Hvad er radius af den inderste cirkulære del af broen?
\item Hvor stort er trædækket af denne bro? 
\end{enumerate}

For en af naboplatformene til den midterste platform følger randen en del af cirklen $C_2$ med ligningen
$$(x+16{,}75)^2+(y-18{,}5)^2=225$$
Cirklerne $C_1$ og $C_2$ skærer hinanden i to punkter $A$ og $B$, som markerer åbningen de to platforme – se figuren.
\begin{enumerate}
\item Find koordinatsættet for hhv. $A$ og $B$.
\item Bestem bredden af åbningen mellem de to platforme (direkte afstand).
\end{enumerate}

\clearpage
\section{Buebro}
\begin{figure}[h!]
\centering
\includegraphics[width=0.8\textwidth]{buebro1.png}
\end{figure}

\begin{wrapfigure}[5]{r}{0.2\textwidth}
\vspace{-18pt}
\includegraphics[width=0.3\textwidth]{buebro2.png}
\end{wrapfigure}
Her ses et billede af en buebro samt en skitse af buen. Koordinatsystemets x-akse flugter med vandoverfladen, og buen er symmetrisk omkring koordinatsystemets $y$-akse.

I en model beskrives vejbanen på broen med funktionen
$$f\left(x\right)=42-0{,}00016\cdot x^2$$
\begin{enumerate}
\item Tegn grafen for vejbanen. Hvor højt kommer vejbanen over havoverfladen?
\end{enumerate}

Buen som bærer vejbanen kan beskrives ved et andengradspolynomium. Broens øverste punkt er 55 m over vandoverfladen og buens bredde ved vandoverfladen er 250 meter.
\begin{enumerate}[resume]
\item Bestem en forskrift for buen og tegn parablen i et koordinatsystem
\item Bestem skæringspunkterne mellem buen og vejbanen
\item Bestem en tilnærmet værdi for længden af kørebanen mellem skæringspunkterne.
\end{enumerate}

En bedre model for vejbanen fås ved at dele den op i tre stykker: et midterstykke mellem de to skæringspunkter, samt et stykke på hver side af skæringerne. Den vandrette længde af det viste billede er 500 meter. I enderne er vejbanen ca. 34 meter over havets overflade.
\begin{enumerate}[resume]
\item Bestem en forskrift for den stykkevist definerede funktion
\end{enumerate}

\end{document}

%%%%%%%%%%%%%%%%%%%%%%%%%%%%%%%%%%%%%%%%%%%%%%%%%%%%%%%%%%%%%%%%%%%%%%%%%%%%%% Skabeloner

% Sidestillet figur
\begin{wrapfigure}[8]{r}{0.2\textwidth}
\vspace{-18pt}
\includegraphics[width=0.3\textwidth]{ovn}
\end{wrapfigure}

% Midterstillet figur
\begin{figure}[h!]
\centering
\includegraphics[width=0.8\textwidth]{abc}
\end{figure}

% Førstillet figur
\
\begin{figure}[h!]
\centering
\vspace{-15pt}
\includegraphics[width=0.35\textwidth]{stud}
\end{figure}

% Tabel
\begin{figure}[h!]
\centering\renewcommand{\arraystretch}{1.5}
\begin{tabularx}{0.9\textwidth}{|l|b|b|b|b|b|b|}
	\hline \cellcolor{hggreen} Decimaltal & 7\% & -51\% & 13,7\% & 126\% & 456\% & 0,28\%\\\hline
	\cellcolor{hggreen} Procenttal &  &  &  &  &  & \\\hline
\end{tabularx}
\end{figure}

% Forklaringsopgaver
\begin{align*}
&&	&						&&\underset{\rule{0.8\linewidth}{0pt}}{\textbf{Forklaring:}}\\[1em]
&&	&\frac{(x+2)^2 - 4}{x} 		&&\underset{\rule{0.8\linewidth}{0.4pt}}{\text{Udtrykket skrives op.}}\\[2em]
&&	=&\ \frac{x^2 + 4 + 4x - 4}{x}	&&\underset{\rule{0.8\linewidth}{0.4pt}}{\text{}}\\[2em]
&&	=&\ \frac{x^2 + 4x}{x}			&&\underset{\rule{0.8\linewidth}{0.4pt}}{\text{}}\\[2em]
&&	=&\ x + 4					&&\underset{\rule{0.8\linewidth}{0.4pt}}{\text{}}\\[2em]
\end{align*}

% Sørg for at figur er placeret før et vidst sted
\FloatBarrier




