\documentclass[10pt,a4paper,table]{article}
	
	\usepackage{amsmath,amssymb,amsthm,mathtools,amsfonts}
	\usepackage{breqn}
	\usepackage[utf8]{inputenc}
	\usepackage[danish]{babel}
	\usepackage[T1]{fontenc}
	
	%%%%%%%%%%%%%%%%%%%%%%%%%%%%%%%%%%%%%%%%%%%%%%%%%%%%%%%%%%%%%%%%%%%%%%%%%%%%%% Fonts
	
	\usepackage{fontspec}
	\setmainfont{Times New Roman}
	\newfontfamily\garamond[Numbers=OldStyle]{EB Garamond}
	\newfontfamily\plantin[Numbers=OldStyle]{Plantin MT Pro}
	\newfontfamily\wal{Walbaum Com}
	\newfontfamily\klav{Klavika}
	\newfontfamily\klavl{Klavika light}
	\newfontfamily\quotefont[Ligatures=TeX]{Linux Libertine O}
	\newfontfamily\vest{TRY Vesterbro Regular}
	\newfontfamily\vestb{TRY Vesterbro Poster}
	\newfontfamily\overp{Overpass}
	
	%%%%%%%%%%%%%%%%%%%%%%%%%%%%%%%%%%%%%%%%%%%%%%%%%%%%%%%%%%%%%%%%%%%%%%%%%%%%%% Colors
	
	\usepackage{xcolor}
	
	\definecolor{frbblue}{RGB}{47, 88, 109}
	\definecolor{hgred}{RGB}{140, 10, 10}
	\definecolor{frbgreenl}{RGB}{162, 202, 137}
	\definecolor{frbgreen}{RGB}{28, 85, 66}
	\definecolor{frbgrey}{RGB}{243, 245, 242}
	
	%%%%%%%%%%%%%%%%%%%%%%%%%%%%%%%%%%%%%%%%%%%%%%%%%%%%%%%%%%%%%%%%%%%%%%%%%%%%%% Other Graphics
	
	\usepackage{graphicx}
	\graphicspath{{../img/}}
	\usepackage{tikz}
	\usepackage{placeins}
	\usepackage{subcaption}
	
	\usepackage[pdfborder={0 0 0}]{hyperref} %% Ingen firkant rundt om referencer

	\usepackage{multicol}
	\usepackage{cellspace}
		\setlength\cellspacetoplimit{100pt}
		\setlength\cellspacebottomlimit{100pt}
	\usepackage{tabularx}
		\newcolumntype{b}{>{\hsize=\hsize}>{\centering\arraybackslash}X}
	\usepackage{siunitx}
	\usepackage{cellspace}
		\setlength\cellspacetoplimit{4pt}
		\setlength\cellspacebottomlimit{4pt}
	
	\usepackage{enumitem}% better controls with enumerating
	\usepackage{wrapfig}
	\usepackage{caption}
	%\usepackage{dirtytalk}
	%\usepackage{tasks}
	\usepackage{titlesec}

	\setlength{\parskip}{0.5\baselineskip}
	\setlength{\parindent}{0cm}
	\setlist[enumerate]{label=\alph*),left=20pt}
	%\setlist[enumerate]{leftmargin=\parindent}
	%\setlist[enumerate]{nosep}
	
	
	%%%%%%%%%%%%%%%%%%%%%%%%%%%%%%%%%%%%%%%%%%%%%%%%%%%%%%%%%%%%%%%%%%%%%%%%%%%%%% New commands
	
	\newcommand\svar[1]{\color{red}{#1}\color{black}}
	
	\usepackage[h]{esvect}
	\newcommand{\twvect}[2]{\ensuremath{\begin{pmatrix}#1\\#2\end{pmatrix}}}
	
	\newcommand*\quotesize{20}
	\newcommand*{\openquote}
		{\tikz[remember picture,overlay,xshift=-3ex,yshift=-0ex]
			\node (OQ) {\quotefont\fontsize{\quotesize}{\quotesize}\selectfont``};\kern0pt}

	\newcommand*{\closequote}[1]
		{\tikz[remember picture,overlay,xshift=2ex,yshift={#1}]
			\node (CQ) {\quotefont\fontsize{\quotesize}{\quotesize}\selectfont''};}
			
	\newcommand{\qm}[1]{``#1''}
	\newcommand{\iquote}[1]{\begin{quote}\itshape \openquote#1\hfill\closequote{0ex} \end{quote}}
	
	%%%%%%%%%%%%%%%%%%%%%%%%%%%%%%%%%%%%%%%%%%%%%%%%%%%%%%%%%%%%%%%%%%%%%%%%%%%%%% Sections
	\usepackage{chngcntr}
	
	\renewcommand\thesection{\arabic{section}}

	\titleformat{\section}[block] 
  		{\flushright\overp\Large\color{frbgreen}}    % format for whole title
  		{}                           % no separate label (empty)
  		{0pt}                        % spacing between label and title
		{DELPRØVE \thesection}       % prepend "Delprøve <number>"
		[{\color{frbgreenl}\vspace{-1em}\hspace*{-0.165\linewidth}\rule{\dimexpr1.165\linewidth\relax}{0.4pt}\kern1mm}]
	
	\counterwithout{subsection}{section}
	\renewcommand\thesubsection{Opgave \arabic{subsection}}
	\titleformat{\subsection}[leftmargin]{\bfseries}{}{0pt}{\hspace{-5em}\thesubsection}
	
	%%%%%%%%%%%%%%%%%%%%%%%%%%%%%%%%%%%%%%%%%%%%%%%%%%%%%%%%%%%%%%%%%%%%%%%%%%%%%% Header & Footer
	
	\usepackage{bigfoot}
	\DeclareNewFootnote{a}
		\renewcommand{\thefootnotea}{\fnsymbol{footnotea}}
			\newcommand{\footnotes}[1]{\footnotea[2]{#1}} % here you can specify what symbol you want in footnotes
			\MakePerPage{footnotes}
	
		\usepackage{fancyhdr}
	\fancypagestyle{plain}{%
		\setlength{\headheight}{60pt}%
		\setlength{\headsep}{0pt}
		\fancyhf{}% No header/footer
		\renewcommand{\footrulewidth}{0pt}% No footer rule
		\renewcommand\headrule{
			\centerline{\begin{minipage}{1\textwidth}
				\color{frbblue}\hrule width \hsize \kern 1mm \hrule width \hsize height 2pt \vspace{4em}
			\end{minipage}}
		}
		\fancyhead[L]{\hspace{0em}\includegraphics[height=2.5em]{frbvuc_logo.png}}
		\fancyhead[R]{\overp\color{frbgreen}\footnotesize Efterår 2025}
		\fancyhead[C]{\overp\color{frbgreenl}\normalsize \overp Mat C-B\\\Large\color{frbgreen}\vestb Aflevering 4\\\footnotesize\color{red}Besvarelse}
		\cfoot{\thepage}
	}
	\pagestyle{fancy}
	\fancyhf{}
	\renewcommand\headrule{
			\centerline{\begin{minipage}{\textwidth}
				\color{frbgreen}\hrule width \hsize \kern 1mm
			\end{minipage}}}
	\lhead{\overp\color{frbgreen}\footnotesize Mat C-B}
	\rhead{\overp\color{frbgreen}\footnotesize Efterår 2025}
	\cfoot{\thepage}
	
	%%%%%%%%%%%%%%%%%%%%%%%%%%%%%%%%%%%%%%%%%%%%%%%%%%%%%%%%%%%%%%%%%%%%%%%%%%%%%% Document

\begin{document}
\thispagestyle{plain}

\section{}
\subsection{}
En cirkel har ligningen
$$(x-1)^2 + (y-4)^2 = 9.$$

\begin{enumerate}
    \item Bestem centrum og radius for cirklen.
\end{enumerate}
\svar{
	Den generelle cirklens ligning for en cirkel med centrum $C(a,b)$ og radius $r$ er givet ved:
	$$(x-a)^2+(y-b)^2=r^2.$$
	Derfor centrum i denne cirkel $C(1,4)$ og radius $r=\sqrt{9}=3$.
}

\begin{enumerate}[resume]
    \item Undersøg, om punktet $P(5,6)$ ligger på cirklen.
\end{enumerate}
\svar{
	Vi indsætter $P(5,6)$ i cirklens ligning:
	$$(5-1)^2+(6-4)^2 = 4^2+2^2=16+4=20.$$
	Da $20 \neq 9$, ligger $P(5,6)$ ikke på cirklen.
}

\subsection{}
Der er givet to punkter $A(1,4)$ og $B(9,10)$.
\begin{enumerate}
    \item Bestem afstanden $|AB|$ mellem de to punkter.
\end{enumerate}
\svar{
	Afstandsformlen mellem to punkter er:
	$$|AB|=\sqrt{(x_2-x_1)^2+(y_2-y_1)^2}.$$
	Her:
	$$|AB|=\sqrt{(9-1)^2+(10-4)^2}=\sqrt{8^2+6^2}=\sqrt{64+36}=\sqrt{100}=10.$$
	Altså er afstanden $|AB|=10$.
}

\subsection{}
En cirkel har centrum i punktet $C(5,-3)$ og har radius $r=4$.
\begin{enumerate}
    \item Bestem en ligning for cirklen.
\end{enumerate}
\svar{
	Den generelle form er:
	$$(x-a)^2+(y-b)^2=r^2.$$
	Her $a=5,\;b=-3,\;r=4$, altså
	$$(x-5)^2+(y+3)^2=16.$$
}

\section{}
\subsection{}
En cirkel er givet ved ligningen
\[
    (x-5)^2 + (y+1)^2 = 13 .
\]

\begin{enumerate}
    \item Bestem en ligning for tangenten $t_1$ til cirklen i punktet $P(2,1)$.
\end{enumerate}
\svar{
    Centrum er $C(5,-1)$ og $P(2,1)$ ligger på cirklen: $(2-5)^2+(1+1)^2=9+4=13$.

    Hældningen på radius $\overline{CP}$ er
    $$a_{CP}=\frac{1-(-1)}{2-5}=\frac{2}{-3}=-\tfrac{2}{3}.$$
    Tangenten er vinkelret på radius, så
    $$a_{t_1}=-\frac{1}{a_{CP}}=\tfrac{3}{2}.$$
    Gennem $P(2,1)$ fås $b$-værdien:
    $$b=y_1-a\cdot x_1=1-\tfrac32\cdot 2=-2$$
    Altså $t_1:\; y=\tfrac{3}{2}x-2$.
}

Cirklen har en anden tangent $t_2$, der er parallel med $t_1$.

\begin{enumerate}
    \setcounter{enumi}{1}
    \item Bestem en ligning for tangenten $t_2$.
\end{enumerate}
\svar{
    Parallel med $t_1$ betyder samme hældning $a=\tfrac{3}{2}$, så vi søger en linje
    $$y=\tfrac{3}{2}x+c,$$
    der er tangent til cirklen. Afstanden fra centrum $C(5,-1)$ til linjen skal være cirklens radius $r=\sqrt{13}$.
    Brug formlen for afstand til $y=ax+b$:
    $$d=\frac{|a x_1+b-y_1|}{\sqrt{a^2+1}}.$$
    Her fås
    $$\frac{\left|\tfrac{3}{2}\cdot 5 + c - (-1)\right|}{\sqrt{\left(\tfrac{3}{2}\right)^2+1}}
      =\sqrt{13}.$$
    Løser man denne ligning i Geogebra får man desværre kun den ene løsning, nemlig vores $b$-værdi fra før:

	\begin{figure}[h!]
		\centering
		\includegraphics[width=0.3\textwidth]{geog}
	\end{figure}
	\FloatBarrier

	Man kan dog tvinge den anden løsning frem ved at gøre tælleren til minus:
	\begin{figure}[h!]
		\centering
		\includegraphics[width=0.3\textwidth]{geog2}
	\end{figure}
	
	Eller løse opgaven grafisk i geogebra.
	Hældningen er derfor $c=-15$.
    Ligningen for den paralelle tangent er derfor:
    $$t_2:\; y=\tfrac{3}{2}x-15.$$
}

\subsection{}
En linje $l$ er givet ved ligningen
\[
    y = 2x + 3 .
\]
Desuden er der givet punktet $P(4,1)$.

\begin{enumerate}
    \item Benyt en formel til at bestemme afstanden fra punktet $P$ til linjen $l$.
\end{enumerate}
\svar{
    Brug afstanden til $y=ax+b$:
    $$d=\frac{|a x_1+b-y_1|}{\sqrt{a^2+1}},\quad a=2,\;b=3,\;(x_1,y_1)=(4,1).$$
    $$d=\frac{|2\cdot 4+3-1|}{\sqrt{2^2+1}}
        =\frac{10}{\sqrt{5}}
        =2\sqrt{5}\approx 4{,}47.$$
}

En cirkel har centrum i $P$ og har radius $5$.

\begin{enumerate}
    \setcounter{enumi}{1}
    \item Bestem koordinatsættet til hvert af skæringspunkterne mellem cirklen og linjen $l$.
\end{enumerate}
\svar{
    Cirklen har ligningen $(x-4)^2+(y-1)^2=25$ og linjen $y=2x+3$.
    Indsæt $y$ i cirklen:
    $$(x-4)^2+(2x+3-1)^2=25
      \;\Rightarrow\;
      (x-4)^2+(2x+2)^2=25.$$
    $$x^2-8x+16+4x^2+8x+4=25
      \;\Rightarrow\;
      5x^2+20=25
      \;\Rightarrow\;
      x^2=1.$$
    Altså $x=\pm 1$, og dermed $y=2x+3$ giver punkterne
    $$(-1,1)\quad \text{og}\quad (1,5).$$
}

\subsection{}
En cirkel $C_1$ har centrum i punktet $(4{,}2\ , 0)$ og har radius $10{,}3$.
\begin{enumerate}
	\item Tegn cirklen $C_1$ i et koordinatsystem.
\end{enumerate}
\svar{
	Vi indtaster cirklens liging i et algebra-vindue i geogebra og får:
	\begin{figure}[h!]
		\centering
		\includegraphics[width=0.8\textwidth]{cirkel}
	\end{figure}	
}

\begin{figure}[h!]
\centering
\includegraphics[width=0.8\textwidth]{vinduesramme}
\caption*{\footnotesize \textit{Billedkilde: 1stdibs.com}}
\end{figure}

På figuren ses en vinduesramme, der er lagt ind i et koordinatsystem. Enheden i koordinatsystemet er decimeter.

I en model afgrænses vinduesrammen af førsteaksen samt cirklen $C_1$ og cirklen $C_2$.

Det oplyses, at cirklen $C_2$ har ligningen
\[
    (x+4{,}2)^2 + y^2 = 10{,}3^2 .
\]

\begin{enumerate}[resume]
    \item Bestem vinduesrammens bredde og højde.
\end{enumerate}
\svar{
	Vi kan få cirklernes skæring med $x$-aksen med Geogebras skæringsværktøj:
	\begin{figure}[h!]
		\centering
		\includegraphics[width=0.3\textwidth]{skæring}
	\end{figure}
	
	Denne skæring sker i $x=6.1$, Bredden må derfor $6.1\cdot 2=12.2$ decimeter. Samme skæringsværktøj kan jeg også bruge til at finde cirklernes skæring med hinanden (eller $y$-aksen).
	Gør man det får man $y=9.4$. Højden må derfor være $9.4$ decimeter.
}

\subsection{}
\
\begin{figure}[h!]
\centering
\vspace{-15pt}
\begin{subfigure}{0.45\textwidth}
    \centering
    \includegraphics[width=\linewidth]{politigaard_foto}
    \caption*{\footnotesize Figur 1}
\end{subfigure}
\hfill
\begin{subfigure}{0.45\textwidth}
    \centering
    \includegraphics[width=\linewidth]{politigaard_plan}
    \caption*{\footnotesize Figur 2}
\end{subfigure}
\end{figure}

Figur 1 viser Københavns Politigård.

På figur 2 er politigårdens grundplan lagt ind i et koordinatsystem, hvor enheden på begge akser er meter. To af de runde bygningsdele er markeret på figuren.

Den store runde gård kan beskrives med cirkel 1, der har ligningen
\[
    (x-54)^2 + (y-34)^2 = 515{,}29 .
\]

\begin{enumerate}
    \item Bestem centrum $C_1$ og radius $r_1$ for cirkel 1.
\end{enumerate}
\svar{
	Centrum er $C_1(54,34)$ og radius $r_1=\sqrt{515.29}=22,7$.
}

Et rundt trappetårn kan beskrives med cirkel 2, der har centrum $C_2(109,106)$ og radius $r_2=5$.

\begin{enumerate}
    \setcounter{enumi}{1}
    \item Bestem den mindste distance mellem de to cirkler.
\end{enumerate}
\svar{
	Den mindste distance må kunne fås ved at finde afstanden mellem de centrummer $C_1$ og $C_2$ og trække deres radier fra denne afstand.
	
	Afstand mellem to punkter:
	$$|C_1C_2|=\sqrt{(109-54)^2+(106-34)^2}\approx90,6$$

	Den mindste afstand:
	$$|C_1C_2|-(r_1+r_2)=90,6-(22,7+5)=62,9$$
	\begin{figure}[h!]
\centering
\includegraphics[width=0.8\textwidth]{cirkler}
\end{figure}
}
\end{document}

%%%%%%%%%%%%%%%%%%%%%%%%%%%%%%%%%%%%%%%%%%%%%%%%%%%%%%%%%%%%%%%%%%%%%%%%%%%%%% Skabeloner

% Sidestillet figur
\begin{wrapfigure}[8]{r}{0.2\textwidth}
\vspace{-18pt}
\includegraphics[width=0.3\textwidth]{ovn}
\end{wrapfigure}

% Midterstillet figur
\begin{figure}[h!]
\centering
\includegraphics[width=0.8\textwidth]{abc}
\end{figure}

% Førstillet figur
\
\begin{figure}[h!]
\centering
\vspace{-15pt}
\includegraphics[width=0.35\textwidth]{stud}
\end{figure}

% Tabel
\begin{figure}[h!]
\centering\renewcommand{\arraystretch}{1.5}
\begin{tabularx}{0.9\textwidth}{|l|b|b|b|b|b|b|}
	\hline \cellcolor{hggreen} Decimaltal & 7\% & -51\% & 13,7\% & 126\% & 456\% & 0,28\%\\\hline
	\cellcolor{hggreen} Procenttal &  &  &  &  &  & \\\hline
\end{tabularx}
\end{figure}

% Forklaringsopgaver
\begin{align*}
&&	&						&&\underset{\rule{0.8\linewidth}{0pt}}{\textbf{Forklaring:}}\\[1em]
&&	&\frac{(x+2)^2 - 4}{x} 		&&\underset{\rule{0.8\linewidth}{0.4pt}}{\text{Udtrykket skrives op.}}\\[2em]
&&	=&\ \frac{x^2 + 4 + 4x - 4}{x}	&&\underset{\rule{0.8\linewidth}{0.4pt}}{\text{}}\\[2em]
&&	=&\ \frac{x^2 + 4x}{x}			&&\underset{\rule{0.8\linewidth}{0.4pt}}{\text{}}\\[2em]
&&	=&\ x + 4					&&\underset{\rule{0.8\linewidth}{0.4pt}}{\text{}}\\[2em]
\end{align*}

% Sørg for at figur er placeret før et vidst sted
\FloatBarrier




