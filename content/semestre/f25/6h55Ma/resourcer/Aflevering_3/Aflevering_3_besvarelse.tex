\documentclass[10pt,a4paper,table]{article}
	
	\usepackage{amsmath,amssymb,amsthm,mathtools,amsfonts}
	\usepackage{breqn}
	\usepackage[utf8]{inputenc}
	\usepackage[danish]{babel}
	\usepackage[T1]{fontenc}
	
	%%%%%%%%%%%%%%%%%%%%%%%%%%%%%%%%%%%%%%%%%%%%%%%%%%%%%%%%%%%%%%%%%%%%%%%%%%%%%% Fonts
	
	\usepackage{fontspec}
	\setmainfont{Times New Roman}
	\newfontfamily\garamond[Numbers=OldStyle]{EB Garamond}
	\newfontfamily\plantin[Numbers=OldStyle]{Plantin MT Pro}
	\newfontfamily\wal{Walbaum Com}
	\newfontfamily\klav{Klavika}
	\newfontfamily\klavl{Klavika light}
	\newfontfamily\quotefont[Ligatures=TeX]{Linux Libertine O}
	\newfontfamily\vest{TRY Vesterbro Regular}
	\newfontfamily\vestb{TRY Vesterbro Poster}
	\newfontfamily\overp{Overpass}
	
	%%%%%%%%%%%%%%%%%%%%%%%%%%%%%%%%%%%%%%%%%%%%%%%%%%%%%%%%%%%%%%%%%%%%%%%%%%%%%% Colors
	
	\usepackage{xcolor}
	
	\definecolor{frbblue}{RGB}{47, 88, 109}
	\definecolor{hgred}{RGB}{140, 10, 10}
	\definecolor{frbgreenl}{RGB}{162, 202, 137}
	\definecolor{frbgreen}{RGB}{28, 85, 66}
	\definecolor{frbgrey}{RGB}{243, 245, 242}
	
	%%%%%%%%%%%%%%%%%%%%%%%%%%%%%%%%%%%%%%%%%%%%%%%%%%%%%%%%%%%%%%%%%%%%%%%%%%%%%% Other Graphics
	
	\usepackage{graphicx}
	\graphicspath{{../img/}}
	\usepackage{tikz}
	\usepackage{placeins}
	
	\usepackage[pdfborder={0 0 0}]{hyperref} %% Ingen firkant rundt om referencer

	\usepackage{multicol}
	\usepackage{cellspace}
		\setlength\cellspacetoplimit{100pt}
		\setlength\cellspacebottomlimit{100pt}
	\usepackage{tabularx}
		\newcolumntype{b}{>{\hsize=\hsize}>{\centering\arraybackslash}X}
	\usepackage{siunitx}
	\usepackage{cellspace}
		\setlength\cellspacetoplimit{4pt}
		\setlength\cellspacebottomlimit{4pt}
	
	\usepackage{enumitem}% better controls with enumerating
	\usepackage{wrapfig}
	\usepackage{caption}
	%\usepackage{dirtytalk}
	%\usepackage{tasks}
	\usepackage{titlesec}

	\setlength{\parskip}{0.5\baselineskip}
	\setlength{\parindent}{0cm}
	\setlist[enumerate]{label=\alph*),left=20pt}
	%\setlist[enumerate]{leftmargin=\parindent}
	%\setlist[enumerate]{nosep}
	
	
	%%%%%%%%%%%%%%%%%%%%%%%%%%%%%%%%%%%%%%%%%%%%%%%%%%%%%%%%%%%%%%%%%%%%%%%%%%%%%% New commands
	
	\newcommand\svar[1]{\color{red}{#1}\color{black}}
	
	\usepackage[h]{esvect}
	\newcommand{\twvect}[2]{\ensuremath{\begin{pmatrix}#1\\#2\end{pmatrix}}}
	
	\newcommand*\quotesize{20}
	\newcommand*{\openquote}
		{\tikz[remember picture,overlay,xshift=-3ex,yshift=-0ex]
			\node (OQ) {\quotefont\fontsize{\quotesize}{\quotesize}\selectfont``};\kern0pt}

	\newcommand*{\closequote}[1]
		{\tikz[remember picture,overlay,xshift=2ex,yshift={#1}]
			\node (CQ) {\quotefont\fontsize{\quotesize}{\quotesize}\selectfont''};}
			
	\newcommand{\qm}[1]{``#1''}
	\newcommand{\iquote}[1]{\begin{quote}\itshape \openquote#1\hfill\closequote{0ex} \end{quote}}
	
	%%%%%%%%%%%%%%%%%%%%%%%%%%%%%%%%%%%%%%%%%%%%%%%%%%%%%%%%%%%%%%%%%%%%%%%%%%%%%% Sections
	\usepackage{chngcntr}
	
	\renewcommand\thesection{\arabic{section}}

	\titleformat{\section}[block] 
  		{\flushright\overp\Large\color{frbgreen}}    % format for whole title
  		{}                           % no separate label (empty)
  		{0pt}                        % spacing between label and title
		{DELPRØVE \thesection}       % prepend "Delprøve <number>"
		[{\color{frbgreenl}\vspace{-1em}\hspace*{-0.165\linewidth}\rule{\dimexpr1.165\linewidth\relax}{0.4pt}\kern1mm}]
	
	\counterwithout{subsection}{section}
	\renewcommand\thesubsection{Opgave \arabic{subsection}}
	\titleformat{\subsection}[leftmargin]{\bfseries}{}{0pt}{\hspace{-5em}\thesubsection}
	
	%%%%%%%%%%%%%%%%%%%%%%%%%%%%%%%%%%%%%%%%%%%%%%%%%%%%%%%%%%%%%%%%%%%%%%%%%%%%%% Header & Footer
	
	\usepackage{bigfoot}
	\DeclareNewFootnote{a}
		\renewcommand{\thefootnotea}{\fnsymbol{footnotea}}
			\newcommand{\footnotes}[1]{\footnotea[2]{#1}} % here you can specify what symbol you want in footnotes
			\MakePerPage{footnotes}
	
		\usepackage{fancyhdr}
	\fancypagestyle{plain}{%
		\setlength{\headheight}{60pt}%
		\setlength{\headsep}{0pt}
		\fancyhf{}% No header/footer
		\renewcommand{\footrulewidth}{0pt}% No footer rule
		\renewcommand\headrule{
			\centerline{\begin{minipage}{1\textwidth}
				\color{frbblue}\hrule width \hsize \kern 1mm \hrule width \hsize height 2pt \vspace{4em}
			\end{minipage}}
		}
		\fancyhead[L]{\hspace{0em}\includegraphics[height=2.5em]{frbvuc_logo.png}}
		\fancyhead[R]{\overp\color{frbgreen}\footnotesize Efterår 2025}
		\fancyhead[C]{\overp\color{frbgreenl}\normalsize \overp Mat C-B\\\Large\color{frbgreen}\vestb Aflevering 3\\\color{red}\footnotesize Besvarelse}
		\cfoot{\thepage}
	}
	\pagestyle{fancy}
	\fancyhf{}
	\renewcommand\headrule{
			\centerline{\begin{minipage}{\textwidth}
				\color{frbgreen}\hrule width \hsize \kern 1mm
			\end{minipage}}}
	\lhead{\overp\color{frbgreen}\footnotesize Mat C-B}
	\rhead{\overp\color{frbgreen}\footnotesize Efterår 2025}
	\cfoot{\thepage}
	
	%%%%%%%%%%%%%%%%%%%%%%%%%%%%%%%%%%%%%%%%%%%%%%%%%%%%%%%%%%%%%%%%%%%%%%%%%%%%%% Document

\begin{document}
\thispagestyle{plain}

\section{}

\subsection{}
Bestem skæringspunktet mellem linjerne
\begin{align*}
2x + y &= 7 \\
x - y &= 1
\end{align*}

\color{red}{%
Fra den anden ligning fås \(y = x - 1\). Indsættes i den første:
\[
2x + (x-1) = 7 \quad \Rightarrow \quad 3x - 1 = 7 \quad \Rightarrow \quad 3x = 8 \quad \Rightarrow \quad x = \tfrac{8}{3}.
\]
\[
y = x - 1 = \tfrac{8}{3} - 1 = \tfrac{5}{3}.
\]
Skæringspunktet er altså \(\left(\tfrac{8}{3}, \tfrac{5}{3}\right)\).
}

\color{black}
\subsection{}
I et koordinatsystem er der givet et punkt \(P(4,1)\) og en linje \(l\) med ligningen
\[
y = \tfrac{3}{4} \cdot x + 2 .
\]

\begin{enumerate}
    \item Bestem afstanden fra punktet \(P\) til linjen \(l\).
\end{enumerate}

\color{red}{%
Linjen har formen \(y = ax + b\) med \(a = \tfrac{3}{4}\) og \(b = 2\). Punktet er \(P(4,1)\).

Formlen for afstanden er:
\[
\text{dist}(P,l) = \frac{|a \cdot x_1 + b - y_1|}{\sqrt{a^2+1}}.
\]

Indsættes værdierne:
\[
\text{dist}(P,l) = \frac{\left|\tfrac{3}{4}\cdot 4 + 2 - 1\right|}{\sqrt{\left(\tfrac{3}{4}\right)^2+1}}
= \frac{|3+2-1|}{\sqrt{\tfrac{9}{16}+1}}
= \frac{4}{\sqrt{\tfrac{25}{16}}}
= \frac{4}{\tfrac{5}{4}}
= \tfrac{16}{5}.
\]

Afstanden er altså \(\tfrac{16}{5} = 3,2\).
}
\color{black}

\subsection{}
Bestem skæringspunktet mellem linjen og parablen
\[
y = 2x + 3 \qquad\text{og}\qquad y = x^2 + x + 1 .
\]

\color{red}
Vi sætter højresiderne lig hinanden:
\begin{align*}
2x+3 &= x^2 + x + 1 && \;\Leftrightarrow\;\\
x^2 - x - 2 &= 0 && \;\Leftrightarrow\;\\
x &= \frac{-b\pm\sqrt{b^2-4ac}}{2a}=\frac{1\pm\sqrt{1-4\cdot 1\cdot (-2)}}{2}=\begin{cases}2\\-1\end{cases}
\end{align*}

$y$-værdien kan vi finde ved at indsætte disse $x$-værdier i linjens ligning: $y=2\cdot 2 + 3 = 7$ og $y=2\cdot (-1) + 3 = 1$.
Skæringspunktet findes således i punkterne $(2,7)$ og $(-1,1)$.
\color{black}

\subsection{}
En linje \(l\) går igennem punkterne \((-4,2)\) og \((6,-3)\).
\begin{enumerate}
    \item Bestem en ligning for linjen \(l\).
\end{enumerate}
\svar{
	Topunktsformlen giver os:
	\begin{align*}	
	a&=\frac{-3-2}{6-(-4)}=\frac{-5}{10}=-\frac{1}{2}\\
	\text{og}\\
	b&=2-(-\frac{1}{2})\cdot (-4)= 0
	\end{align*}
	I alt altså $y=-\tfrac12x$
}

Linjen \(m\) er givet ved ligningen
$$-3x+4y+18=0$$
\begin{enumerate}[resume]
    \item Bestem skæringspunktet mellem de to linjer \(l\) og \(m\).
\end{enumerate}
\svar{
	Indsæt $y$ fra $l$ i $m$:
	\begin{align*}
		-3x+4(-\tfrac12x)+18&=0\\
		-5x+18&=0\\
		5x&=18\\
		x&=\frac{18}{5}
	\end{align*}
	og indsæt denne $x$-værdi i $l$: $y=-\tfrac12\cdot \tfrac{18}{5}=-\tfrac95$.
	Skæringspunktet ligger alså i $\left(\tfrac{18}{5},-\tfrac95\right)$.
}

\begin{enumerate}[resume]
    \item Bestem den spidse vinkel mellem \(l\) og \(m\).
\end{enumerate}

\svar{
	Vinklen mellem de to linjer kan findes ved at kigge på de to hældninger. Den kan dog ikke findes uden hjælpemidler, beklager fejlen fra min side.
	Hældningen for linjen $l$ er $a=-\tfrac12$, mens hældningen for $m$ findes ved at isolere y i $m$:
	\begin{align*}
		-3x+4y+18&=0\\
		4y&=3x-18\\
		y&=\tfrac34x -\tfrac{18}{4}
	\end{align*}

	Altså er hældningen for $m$, $c=\tfrac34$. Vinklen fås så ved at isolere hældningsvinklerne i formlen:
	$$\tan(v)=a$$
	og trække dem fra hinanden:
	$$\tan^{-1}\left(\tfrac34\right)-\tan^{-1}\left(\tfrac12\right)=63,4349488229^\circ$$
}

\section{}

\subsection{}
En parabel har ligningen \(y=2x^2-20x+51\).
\begin{enumerate}
    \item Benyt en formel til at bestemme koordinatsættet til parablens toppunkt.
\end{enumerate}

\svar{
	En parabels toppunkt har $x$- og $y$-koordinaterne:
	$$T_x=-\frac{b}{2a} \quad \text{og} \quad T_y=\frac{-d}{4a}$$
	Her er $a=2$, $b=-20$, $c=51$ og $d=b^2-4ac=(-20)^2-4\cdot 2\cdot 51=-8$. Derfor får vi:
	$$T_x=\frac{20}{2\cdot 2}=5 \quad \text{og} \quad T_y\frac{-(-8)}{4\cdot 2}=1$$
}

En ret linje har ligningen \(y=-0{,}4x+4{,}6\).
\begin{enumerate}\setcounter{enumi}{1}
    \item Benyt en formel til at bestemme afstanden fra parablens toppunkt til linjen.
\end{enumerate}
\svar{
	Afstanden mellem et punkt og en linje fås ved:
$$\text{dist}(P,l)=\frac{|a x_1+b-y_1|}{\sqrt{a^2+1}}$$
Her har vi $a=-0{,}4,\; b=4{,}6,\; (x_1,y_1)=(5,1)$ som vi indsætter i formlen:
$$\text{dist}(P,l)=\frac{|(-0{,}4)\cdot 5+4{,}6-1|}{\sqrt{(-0{,}4)^2+1}}
=\frac{1{,}6}{\sqrt{1{,}16}}
\approx 1{,}49.
$$
}



\subsection{}
En linje \(l\) er givet ved ligningen \(y=3x+7\).
\begin{enumerate}
    \item Benyt en formel til at bestemme afstanden fra punktet \(P(2,4)\) til linjen \(l\).
\end{enumerate}

\color{red}{%
Afstandsformlen for \(y=ax+b\):
$$
\text{dist}(P,l)=\frac{|a x_1+b-y_1|}{\sqrt{a^2+1}}$$
med $a=3,\; b=7,\; (x_1,y_1)=(2,4)$:
$$\text{dist}(P,l)=\frac{|3\cdot 2+7-4|}{\sqrt{3^2+1}}
=\frac{9}{\sqrt{10}}
\approx 2{,}85.
$$
}
\end{document}

%%%%%%%%%%%%%%%%%%%%%%%%%%%%%%%%%%%%%%%%%%%%%%%%%%%%%%%%%%%%%%%%%%%%%%%%%%%%%% Skabeloner

% Sidestillet figur
\begin{wrapfigure}[8]{r}{0.2\textwidth}
\vspace{-18pt}
\includegraphics[width=0.3\textwidth]{ovn}
\end{wrapfigure}

% Midterstillet figur
\begin{figure}[h!]
\centering
\includegraphics[width=0.8\textwidth]{abc}
\end{figure}

% Førstillet figur
\
\begin{figure}[h!]
\centering
\vspace{-15pt}
\includegraphics[width=0.35\textwidth]{stud}
\end{figure}

% Tabel
\begin{figure}[h!]
\centering\renewcommand{\arraystretch}{1.5}
\begin{tabularx}{0.9\textwidth}{|l|b|b|b|b|b|b|}
	\hline \cellcolor{hggreen} Decimaltal & 7\% & -51\% & 13,7\% & 126\% & 456\% & 0,28\%\\\hline
	\cellcolor{hggreen} Procenttal &  &  &  &  &  & \\\hline
\end{tabularx}
\end{figure}

% Forklaringsopgaver
\begin{align*}
&&	&						&&\underset{\rule{0.8\linewidth}{0pt}}{\textbf{Forklaring:}}\\[1em]
&&	&\frac{(x+2)^2 - 4}{x} 		&&\underset{\rule{0.8\linewidth}{0.4pt}}{\text{Udtrykket skrives op.}}\\[2em]
&&	=&\ \frac{x^2 + 4 + 4x - 4}{x}	&&\underset{\rule{0.8\linewidth}{0.4pt}}{\text{}}\\[2em]
&&	=&\ \frac{x^2 + 4x}{x}			&&\underset{\rule{0.8\linewidth}{0.4pt}}{\text{}}\\[2em]
&&	=&\ x + 4					&&\underset{\rule{0.8\linewidth}{0.4pt}}{\text{}}\\[2em]
\end{align*}

% Sørg for at figur er placeret før et vidst sted
\FloatBarrier




