\documentclass[10pt,a4paper,table]{article}
	
	\usepackage{amsmath,amssymb,amsthm,mathtools,amsfonts}
	\usepackage{breqn}
	\usepackage[utf8]{inputenc}
	\usepackage[danish]{babel}
	\usepackage[T1]{fontenc}
	
	%%%%%%%%%%%%%%%%%%%%%%%%%%%%%%%%%%%%%%%%%%%%%%%%%%%%%%%%%%%%%%%%%%%%%%%%%%%%%% Fonts
	
	\usepackage{fontspec}
	\setmainfont{Times New Roman}
	\newfontfamily\garamond[Numbers=OldStyle]{EB Garamond}
	\newfontfamily\plantin[Numbers=OldStyle]{Plantin MT Pro}
	\newfontfamily\wal{Walbaum Com}
	\newfontfamily\klav{Klavika}
	\newfontfamily\klavl{Klavika light}
	\newfontfamily\quotefont[Ligatures=TeX]{Linux Libertine O}
	\newfontfamily\vest{TRY Vesterbro Regular}
	\newfontfamily\vestb{TRY Vesterbro Poster}
	\newfontfamily\overp{Overpass}
	
	%%%%%%%%%%%%%%%%%%%%%%%%%%%%%%%%%%%%%%%%%%%%%%%%%%%%%%%%%%%%%%%%%%%%%%%%%%%%%% Colors
	
	\usepackage{xcolor}
	\usepackage{transparent}
	
	\definecolor{frbblue}{RGB}{47, 88, 109}
	\definecolor{hgred}{RGB}{140, 10, 10}
	\definecolor{frbgreenl}{RGB}{162, 202, 137}
	\definecolor{frbgreen}{RGB}{28, 85, 66}
	\definecolor{frbgrey}{RGB}{243, 245, 242}
	
	%%%%%%%%%%%%%%%%%%%%%%%%%%%%%%%%%%%%%%%%%%%%%%%%%%%%%%%%%%%%%%%%%%%%%%%%%%%%%% Other Graphics
	
	\usepackage{graphicx}
	\graphicspath{{../img/}}
	\usepackage{tikz}
	\usetikzlibrary{calc}
	\usepackage{placeins}
	
	\usepackage[pdfborder={0 0 0}]{hyperref} %% Ingen firkant rundt om referencer

	\usepackage{multicol}
	\usepackage{cellspace}
		\setlength\cellspacetoplimit{100pt}
		\setlength\cellspacebottomlimit{100pt}
	\usepackage{tabularx}
		\newcolumntype{b}{>{\hsize=\hsize}>{\centering\arraybackslash}X}
	\usepackage{siunitx}
	\usepackage{cellspace}
		\setlength\cellspacetoplimit{4pt}
		\setlength\cellspacebottomlimit{4pt}
	
	\usepackage{enumitem}% better controls with enumerating
	\usepackage{wrapfig}
	\usepackage{caption}
	%\usepackage{dirtytalk}
	%\usepackage{tasks}
	\usepackage{titlesec}

	\setlength{\parskip}{0.5\baselineskip}
	\setlength{\parindent}{0cm}
	\setlist[enumerate]{label=\alph*),left=20pt}
	%\setlist[enumerate]{leftmargin=\parindent}
	%\setlist[enumerate]{nosep}
	
	
	%%%%%%%%%%%%%%%%%%%%%%%%%%%%%%%%%%%%%%%%%%%%%%%%%%%%%%%%%%%%%%%%%%%%%%%%%%%%%% New commands
	
	\newcommand\svar[1]{\newline\textcolor{red}{\textbf{Svar}\\#1}}
	
	\usepackage[h]{esvect}
	\newcommand{\twvect}[2]{\ensuremath{\begin{pmatrix}#1\\#2\end{pmatrix}}}
	
	\newcommand*\quotesize{20}
	\newcommand*{\openquote}
		{\tikz[remember picture,overlay,xshift=-3ex,yshift=-0ex]
			\node (OQ) {\quotefont\fontsize{\quotesize}{\quotesize}\selectfont``};\kern0pt}

	\newcommand*{\closequote}[1]
		{\tikz[remember picture,overlay,xshift=2ex,yshift={#1}]
			\node (CQ) {\quotefont\fontsize{\quotesize}{\quotesize}\selectfont''};}
			
	\newcommand{\qm}[1]{``#1''}
	\newcommand{\iquote}[1]{\begin{quote}\itshape \openquote#1\hfill\closequote{0ex} \end{quote}}
	
	%%%%%%%%%%%%%%%%%%%%%%%%%%%%%%%%%%%%%%%%%%%%%%%%%%%%%%%%%%%%%%%%%%%%%%%%%%%%%% Sections
	\usepackage{chngcntr}
	
	\renewcommand\thesection{\arabic{section}}

	\titleformat{\section}[block] 
  		{\flushleft\overp\Large\color{frbgreen}}    % format for whole title
  		{}                           % no separate label (empty)
  		{0pt}                        % spacing between label and title
		{\hspace{-6em}\begin{minipage}[t]{0.2\textwidth}\raggedleft DEL \thesection\\\footnotesize\vest}       % prepend "Delprøve <number>"
		[\end{minipage}\vspace{-2.9em}]
	
	\counterwithout{subsection}{section}
	\renewcommand\thesubsection{Øvelse \arabic{subsection}}
	\titleformat{\subsection}[leftmargin]{\bfseries}{}{0pt}{\hspace{-5em}\thesubsection}
	
	%%%%%%%%%%%%%%%%%%%%%%%%%%%%%%%%%%%%%%%%%%%%%%%%%%%%%%%%%%%%%%%%%%%%%%%%%%%%%% Header & Footer
	
	\usepackage{bigfoot}
	\DeclareNewFootnote{a}
		\renewcommand{\thefootnotea}{\fnsymbol{footnotea}}
			\newcommand{\footnotes}[1]{\footnotea[2]{#1}} % here you can specify what symbol you want in footnotes
			\MakePerPage{footnotes}
	
		\usepackage{fancyhdr}
	\fancypagestyle{plain}{%
		\setlength{\headheight}{60pt}%
		\setlength{\headsep}{0pt}
		\fancyhf{}% No header/footer
		\renewcommand{\footrulewidth}{0pt}% No footer rule
		\renewcommand\headrule{
			\centerline{\begin{minipage}{1\textwidth}
				\color{frbblue}\hrule width \hsize \kern 1mm
			\end{minipage}}
		}
		\fancyhead[L]{\hspace{0em}\includegraphics[height=2.5em]{frbvuc_logo.png}}
		\fancyhead[R]{\overp\color{frbgreen}\footnotesize Efterår 2025}
		\fancyhead[C]{\overp\color{frbgreenl}\normalsize \overp Mat C-B\\\Large\color{frbgreen}\vestb Projekt\\\normalsize\overp Held i spil}
		\cfoot{\thepage}
	}
	\pagestyle{fancy}
	\fancyhf{}
	\renewcommand\headrule{
			\centerline{\begin{minipage}{\textwidth}
				\color{frbgreen}\hrule width \hsize \kern 1mm
			\end{minipage}}}
	\lhead{\overp\color{frbgreen}\footnotesize Mat C-B}
	\rhead{\overp\color{frbgreen}\footnotesize Efterår 2025}
	\cfoot{\thepage}
	
	%%%%%%%%%%%%%%%%%%%%%%%%%%%%%%%%%%%%%%%%%%%%%%%%%%%%%%%%%%%%%%%%%%%%%%%%%%%%%% Document

\begin{document}
\thispagestyle{plain}
\begin{tikzpicture}[remember picture,overlay]
  % a light grey/green like in your screenshot
  \definecolor{bgline}{RGB}{205,214,208}

  % center for the circles (placed a bit *outside* the page to the left)
  \coordinate (C) at ($(current page.north)+(25mm,35mm)$);

  % two thin concentric circles
  \draw[frbblue, line width=0.4pt, opacity=0.3] (C) circle (95mm);
\end{tikzpicture}

\vspace{1em}


\section{Terningekast}
\begin{figure}[h!]
\centering
\includegraphics[width=0.25\textwidth]{terning}
\includegraphics[width=0.25\textwidth]{terning_grøn.jpg}
\end{figure}

Der kastes med to 4-sidede terninger. I et spil betragtes summen af de to terningers udfald.

\begin{enumerate}
    \item Udfyld denne tabel, der beskriver udfaldsrummet:
\end{enumerate}

\begin{figure}[h!]
\centering
\includegraphics[width=0.65\textwidth]{Terningetabel.png}
\end{figure}

\begin{enumerate}
    \item Udregn sandsynlighederne for de forskellige udfald.
    
    \item Hvad er sandsynligheden for at få en sum på 5 eller større?
    
    \item Hvad er sandsynligheden for, at man slår 1 med den grønne terning og et lige tal med den blå terning?
\end{enumerate}

\clearpage
\section{Gevinster}

I et bestemt hasardspil spiller du om penge. Du begynder med 100 kr og vinder i gennemsnit 2,50 kr pr spil.

\begin{enumerate}
    \item[a)] Opstil en funktion $f(x)$, hvor $x$ er antallet af afviklede spil og $f(x)$ er de penge, som du har efter $x$ spil.
    \item[b)] Hvor mange spil skal der spilles før du har 180 kr?
\end{enumerate}

Formuen for en spiller af et bestemt spil kan beskrives med funktionen:
$$
g(x) = -x^4 + 12x^3 + 10x^2 - 60x + 100, \qquad 0 \le x \le 12
$$
hvor $x$ er antallet af minutters spil og $g(x)$ er spillerens formue.

\begin{enumerate}
    \item[c)] Tegn grafen for $g(x)$ i GeoGebra.
    \item[d)] Til hvilke tidspunkter under spillet er formuen for spilleren præcis 2000 kr?
    \item[e)] Efter hvor mange minutters spil har spilleren den største formue?
    \item[f)] Udregn $g'(5)$ og forklar, hvad dette tal fortæller om spillerens formue.
    \item[g)] Udregn $g(6) - g(5)$ og sammenlign dette resultat med svaret i 2f).
    \item[h)] På hvilket tidspunkt i spillet vokser spillerens formue hurtigst?
\end{enumerate}


\clearpage
\section{Ærlige mønter?}
\begin{wrapfigure}[11]{r}{0.2\textwidth}
\vspace{-18pt}
\includegraphics[width=0.4\textwidth]{cointoss.jpeg}
\end{wrapfigure}

En special mønt er lavet, så sandsynligheden for plat er 0,4 og sandsynligheden for krone er 0,6.  
I et eksperiment kastes mønten 25 gange.

\begin{enumerate}
    \item[a)] Gør rede for, at det er et binomialfordelt eksperiment.
    \item[b)] Beregn sandsynligheden for, at der kastes krone netop 15 gange.
    \item[c)] Beregn sandsynligheden for, at der kastes plat højst 12 gange.
\end{enumerate}

Den specielle mønt blandes op i en skål med en masse almindelige (ærlige) mønter.  
Der tages en tilfældig mønt op af skålen. Mønten kastes 25 gange.  
15 af de 25 kast bliver plat.

\begin{enumerate}
    \item[d)] Undersøg, om mønten er den specielle mønt.
\end{enumerate}

\end{document}

%%%%%%%%%%%%%%%%%%%%%%%%%%%%%%%%%%%%%%%%%%%%%%%%%%%%%%%%%%%%%%%%%%%%%%%%%%%%%% Skabeloner

% Sidestillet figur
\begin{wrapfigure}[8]{r}{0.2\textwidth}
\vspace{-18pt}
\includegraphics[width=0.3\textwidth]{ovn}
\end{wrapfigure}

% Midterstillet figur
\begin{figure}[h!]
\centering
\includegraphics[width=0.8\textwidth]{abc}
\end{figure}

% Førstillet figur
\
\begin{figure}[h!]
\centering
\vspace{-15pt}
\includegraphics[width=0.35\textwidth]{stud}
\end{figure}

% Tabel
\begin{figure}[h!]
\centering\renewcommand{\arraystretch}{1.5}
\begin{tabularx}{0.9\textwidth}{|l|b|b|b|b|b|b|}
	\hline \cellcolor{hggreen} Decimaltal & 7\% & -51\% & 13,7\% & 126\% & 456\% & 0,28\%\\\hline
	\cellcolor{hggreen} Procenttal &  &  &  &  &  & \\\hline
\end{tabularx}
\end{figure}

% Forklaringsopgaver
\begin{align*}
&&	&						&&\underset{\rule{0.8\linewidth}{0pt}}{\textbf{Forklaring:}}\\[1em]
&&	&\frac{(x+2)^2 - 4}{x} 		&&\underset{\rule{0.8\linewidth}{0.4pt}}{\text{Udtrykket skrives op.}}\\[2em]
&&	=&\ \frac{x^2 + 4 + 4x - 4}{x}	&&\underset{\rule{0.8\linewidth}{0.4pt}}{\text{}}\\[2em]
&&	=&\ \frac{x^2 + 4x}{x}			&&\underset{\rule{0.8\linewidth}{0.4pt}}{\text{}}\\[2em]
&&	=&\ x + 4					&&\underset{\rule{0.8\linewidth}{0.4pt}}{\text{}}\\[2em]
\end{align*}

% Sørg for at figur er placeret før et vidst sted
\FloatBarrier




