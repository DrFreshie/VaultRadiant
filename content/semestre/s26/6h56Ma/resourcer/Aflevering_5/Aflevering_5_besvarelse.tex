\documentclass[10pt,a4paper,table]{article}
	
	\usepackage{amsmath,amssymb,amsthm,mathtools,amsfonts}
	\usepackage{breqn}
	\usepackage[utf8]{inputenc}
	\usepackage[danish]{babel}
	\usepackage[T1]{fontenc}
	
	%%%%%%%%%%%%%%%%%%%%%%%%%%%%%%%%%%%%%%%%%%%%%%%%%%%%%%%%%%%%%%%%%%%%%%%%%%%%%% Fonts
	
	\usepackage{fontspec}
	\setmainfont{Times New Roman}
	\newfontfamily\garamond[Numbers=OldStyle]{EB Garamond}
	\newfontfamily\plantin[Numbers=OldStyle]{Plantin MT Pro}
	\newfontfamily\wal{Walbaum Com}
	\newfontfamily\klav{Klavika}
	\newfontfamily\klavl{Klavika light}
	\newfontfamily\quotefont[Ligatures=TeX]{Linux Libertine O}
	\newfontfamily\vest{TRY Vesterbro Regular}
	\newfontfamily\vestb{TRY Vesterbro Poster}
	\newfontfamily\overp{Overpass}
	
	%%%%%%%%%%%%%%%%%%%%%%%%%%%%%%%%%%%%%%%%%%%%%%%%%%%%%%%%%%%%%%%%%%%%%%%%%%%%%% Colors
	
	\usepackage{xcolor}
	
	\definecolor{frbblue}{RGB}{47, 88, 109}
	\definecolor{hgred}{RGB}{140, 10, 10}
	\definecolor{frbgreenl}{RGB}{162, 202, 137}
	\definecolor{frbgreen}{RGB}{28, 85, 66}
	\definecolor{frbgrey}{RGB}{243, 245, 242}
	
	%%%%%%%%%%%%%%%%%%%%%%%%%%%%%%%%%%%%%%%%%%%%%%%%%%%%%%%%%%%%%%%%%%%%%%%%%%%%%% Other Graphics
	
	\usepackage{graphicx}
	\graphicspath{{../img/}}
	\usepackage{tikz}
	\usepackage{placeins}
	\usepackage{subcaption}
	
	\usepackage[pdfborder={0 0 0}]{hyperref} %% Ingen firkant rundt om referencer

	\usepackage{multicol}
	\usepackage{cellspace}
		\setlength\cellspacetoplimit{100pt}
		\setlength\cellspacebottomlimit{100pt}
	\usepackage{tabularx}
		\newcolumntype{b}{>{\hsize=\hsize}>{\centering\arraybackslash}X}
	\usepackage{siunitx}
	\usepackage{cellspace}
		\setlength\cellspacetoplimit{4pt}
		\setlength\cellspacebottomlimit{4pt}
	
	\usepackage{enumitem}% better controls with enumerating
	\usepackage{wrapfig}
	\usepackage{caption}
	%\usepackage{dirtytalk}
	%\usepackage{tasks}
	\usepackage{titlesec}

	\setlength{\parskip}{0.5\baselineskip}
	\setlength{\parindent}{0cm}
	\setlist[enumerate]{label=\alph*),left=20pt}
	%\setlist[enumerate]{leftmargin=\parindent}
	%\setlist[enumerate]{nosep}
	
	
	%%%%%%%%%%%%%%%%%%%%%%%%%%%%%%%%%%%%%%%%%%%%%%%%%%%%%%%%%%%%%%%%%%%%%%%%%%%%%% New commands
	
	\newcommand\svar[1]{\color{red}{#1}\color{black}}
	
	\usepackage[h]{esvect}
	\newcommand{\twvect}[2]{\ensuremath{\begin{pmatrix}#1\\#2\end{pmatrix}}}
	
	\newcommand*\quotesize{20}
	\newcommand*{\openquote}
		{\tikz[remember picture,overlay,xshift=-3ex,yshift=-0ex]
			\node (OQ) {\quotefont\fontsize{\quotesize}{\quotesize}\selectfont``};\kern0pt}

	\newcommand*{\closequote}[1]
		{\tikz[remember picture,overlay,xshift=2ex,yshift={#1}]
			\node (CQ) {\quotefont\fontsize{\quotesize}{\quotesize}\selectfont''};}
			
	\newcommand{\qm}[1]{``#1''}
	\newcommand{\iquote}[1]{\begin{quote}\itshape \openquote#1\hfill\closequote{0ex} \end{quote}}
	
	%%%%%%%%%%%%%%%%%%%%%%%%%%%%%%%%%%%%%%%%%%%%%%%%%%%%%%%%%%%%%%%%%%%%%%%%%%%%%% Sections
	\usepackage{chngcntr}
	
	\renewcommand\thesection{\arabic{section}}

	\titleformat{\section}[block] 
  		{\flushright\overp\Large\color{frbgreen}}    % format for whole title
  		{}                           % no separate label (empty)
  		{0pt}                        % spacing between label and title
		{DELPRØVE \thesection}       % prepend "Delprøve <number>"
		[{\color{frbgreenl}\vspace{-1em}\hspace*{-0.165\linewidth}\rule{\dimexpr1.165\linewidth\relax}{0.4pt}\kern1mm}]
	
	\counterwithout{subsection}{section}
	\renewcommand\thesubsection{Opgave \arabic{subsection}}
	\titleformat{\subsection}[leftmargin]{\bfseries}{}{0pt}{\hspace{-5em}\thesubsection}
	
	%%%%%%%%%%%%%%%%%%%%%%%%%%%%%%%%%%%%%%%%%%%%%%%%%%%%%%%%%%%%%%%%%%%%%%%%%%%%%% Header & Footer
	
	\usepackage{bigfoot}
	\DeclareNewFootnote{a}
		\renewcommand{\thefootnotea}{\fnsymbol{footnotea}}
			\newcommand{\footnotes}[1]{\footnotea[2]{#1}} % here you can specify what symbol you want in footnotes
			\MakePerPage{footnotes}
	
		\usepackage{fancyhdr}
	\fancypagestyle{plain}{%
		\setlength{\headheight}{60pt}%
		\setlength{\headsep}{0pt}
		\fancyhf{}% No header/footer
		\renewcommand{\footrulewidth}{0pt}% No footer rule
		\renewcommand\headrule{
			\centerline{\begin{minipage}{1\textwidth}
				\color{frbblue}\hrule width \hsize \kern 1mm \hrule width \hsize height 2pt \vspace{4em}
			\end{minipage}}
		}
		\fancyhead[L]{\hspace{0em}\includegraphics[height=2.5em]{frbvuc_logo.png}}
		\fancyhead[R]{\overp\color{frbgreen}\footnotesize Efterår 2025}
		\fancyhead[C]{\overp\color{frbgreenl}\normalsize \overp Mat C-B\\\Large\color{frbgreen}\vestb Aflevering 5}
		\cfoot{\thepage}
	}
	\pagestyle{fancy}
	\fancyhf{}
	\renewcommand\headrule{
			\centerline{\begin{minipage}{\textwidth}
				\color{frbgreen}\hrule width \hsize \kern 1mm
			\end{minipage}}}
	\lhead{\overp\color{frbgreen}\footnotesize Mat C-B}
	\rhead{\overp\color{frbgreen}\footnotesize Efterår 2025}
	\cfoot{\thepage}
	
	%%%%%%%%%%%%%%%%%%%%%%%%%%%%%%%%%%%%%%%%%%%%%%%%%%%%%%%%%%%%%%%%%%%%%%%%%%%%%% Document

\begin{document}
\thispagestyle{plain}

\section{}
\subsection{}
Funktionen $f$ er givet ved
\[
    f(x) = x^5 - 7x^2 .
\]

\begin{enumerate}
    \item Bestem $f'(x)$.
\end{enumerate}
\svar{
	$$
	f'(x)=5x^4-7\cdot 2x^1=5x^4-14x
	$$
}

\subsection{}
En funktion $f$ er givet ved
\[
    f(x) = x^3 .
\]

\begin{enumerate}
    \item Bestem $f'(x)$.
\end{enumerate}
\svar{
	$$f'(x)=3x^2$$
}

\subsection{}
Om funktionen $f$ oplyses følgende
\begin{align*}
    f(2)&=3\\
    f'(2)&=-1 .
\end{align*}

Netop én af de nedenstående figurer viser grafen for $f$.

\begin{enumerate}
    \item Forklar, hvilken af de tre figurer der viser grafen for $f$, og forklar, hvorfor det ikke kan være de to andre.
\end{enumerate}

\begin{figure}[h!]
\centering
\includegraphics[width=0.8\textwidth]{grafer3}
\end{figure}

\svar{
	Da hældningen for tangenten i $x=2$ er negativ, må parablen være faldende her, og det kan derfor ikke være figur 3.
	Af de sidste to figurer er det kun figur 2 der går gennem punktet $(2,3)$, hvorfor det må være grafen for $f$.
}

\subsection{}
For en differentiabel funktion $f$ er nulpunkter og fortegn for $f'(x)$ angivet på tallinjen nedenfor.

\begin{figure}[h!]
\centering
\includegraphics[width=0.8\textwidth]{fortegn}
\end{figure}
\FloatBarrier

Det oplyses, at netop én af de tre figurer viser grafen for $f$.

\begin{figure}[h!]
\centering
\includegraphics[width=0.8\textwidth]{grafer}
\end{figure}

\begin{enumerate}
    \item Hvilken af de tre figurer viser grafen for $f$? Begrund svaret.
\end{enumerate}
\svar{
	Kun figur 2 og 3 har vendepunkter i $x=1$ og $x=3$, og af disse er det kun figur 2 som er faldende i intervallet $x<1$, hvorfor denne figur må vise grafen for $f$.
}

\section{}
\subsection{}
En funktion $f$ har forskriften
\[
    f(x)=\tfrac{1}{3}x^{3}-\tfrac{7}{2}x^{2}+10x-3 .
\]

\begin{enumerate}
    \item Bestem $f'(x)$, og løs ligningen $f'(x)=0$.
\end{enumerate}
\svar{
	I det følgende tre opgaver vil jeg finde differentialkvotienten ved hjælp af $f'$-knappen i Geogebras CAS-vindue.
	Ligningerne vil blive løst med $x\approx$-knappen. Først de afledte funktion:
	$$f'(x)=x^2-7x+10$$
	Så til ligningen:
	\begin{align*}
		f'(x)&=0\\
		x^2-7x+10&=0\\
		x&=\begin{cases}2\\5\end{cases}
	\end{align*}
}

\begin{enumerate}[resume]
    \item Bestem monotoniforholdene for $f$.
\end{enumerate}
\svar{
	Funktionen har altså vendetangenter i $x=2$ og $x=5$. Hældningen rundt om disse punkter findes først:
	$$f'(1)=4, \quad f'(3)=-2, \quad f'(6)=4$$

	Herudfra ved vi så at funktionen er voksende i intervallene $[-\infty ; 2]$ og $[5;\infty]$, og aftagende i $[2;5]$.
}

\subsection{}
Der er givet funktionen
\[
    f(x)=\frac{x^{3}-2x^{2}}{x^{2}+1}\, .
\]

\begin{enumerate}
    \item Bestem $f'(x)$.
\end{enumerate}
\svar{
	$$f'(x)=\frac{-2x(x^3-2x^2)+(3x^2-4x)(x^2+1)}{(x^2+1)^2}$$
}

\begin{enumerate}[resume]
    \item Løs ligningen $f'(x)=0$, og bestem monotoniforholdene for $f$.
\end{enumerate}
\svar{
	\begin{align*}
	f'(x)&=0\\
	x&=\begin{cases}0\\1\end{cases}
	\end{align*}
	Funktionen har altså vendetangenter i $x=0$ og $x=1$. Hældningen rundt om disse punkter findes først:
	$$f'(-1)=2, \quad f'(0.5)=-0.76, \quad f'(2)=\tfrac45$$

	Herudfra ved vi så at funktionen er voksende i intervallene $[-\infty ; 0]$ og $[1;\infty]$, og aftagende i $[0;1]$.
}

\subsection{}
En funktion $f$ er givet ved
$$
    f(x) = \frac{12x^5 - 45x^4 + 40x^3 + 500}{360}.
$$

\begin{enumerate}
    \item Bestem $f'(x)$.
\end{enumerate}
\svar{
	$$f'(x)=\tfrac16x^4-\tfrac12x^3+\tfrac13x^2$$
}

\begin{enumerate}[resume]
    \item Gør ved hjælp af $f'(x)$ rede for, at $f$ er aftagende i intervallet fra $x=1$ til $x=2$.
\end{enumerate}
\svar{
	Vi finder først funktionens vendepunkter:
	\begin{align*}
	f'(x)&=0\\
	x&=\begin{cases}0\\1\\2\end{cases}
	\end{align*}
	Funktionen vender altså i $x=1$ og i $x=2$, og er altså enten voksende eller aftagende her imellem.
	Udregner vi hældningen et tilfældigt sted i intervallet:
	$$f'(1.5)=-0.09,$$
	kan vi se at funktionen rigtig nok er aftagende i intervallet.
}

\end{document}

%%%%%%%%%%%%%%%%%%%%%%%%%%%%%%%%%%%%%%%%%%%%%%%%%%%%%%%%%%%%%%%%%%%%%%%%%%%%%% Skabeloner

% Sidestillet figur
\begin{wrapfigure}[8]{r}{0.2\textwidth}
\vspace{-18pt}
\includegraphics[width=0.3\textwidth]{ovn}
\end{wrapfigure}

% Midterstillet figur
\begin{figure}[h!]
\centering
\includegraphics[width=0.8\textwidth]{abc}
\end{figure}

% Førstillet figur
\
\begin{figure}[h!]
\centering
\vspace{-15pt}
\includegraphics[width=0.35\textwidth]{stud}
\end{figure}

% Tabel
\begin{figure}[h!]
\centering\renewcommand{\arraystretch}{1.5}
\begin{tabularx}{0.9\textwidth}{|l|b|b|b|b|b|b|}
	\hline \cellcolor{hggreen} Decimaltal & 7\% & -51\% & 13,7\% & 126\% & 456\% & 0,28\%\\\hline
	\cellcolor{hggreen} Procenttal &  &  &  &  &  & \\\hline
\end{tabularx}
\end{figure}

% Forklaringsopgaver
\begin{align*}
&&	&						&&\underset{\rule{0.8\linewidth}{0pt}}{\textbf{Forklaring:}}\\[1em]
&&	&\frac{(x+2)^2 - 4}{x} 		&&\underset{\rule{0.8\linewidth}{0.4pt}}{\text{Udtrykket skrives op.}}\\[2em]
&&	=&\ \frac{x^2 + 4 + 4x - 4}{x}	&&\underset{\rule{0.8\linewidth}{0.4pt}}{\text{}}\\[2em]
&&	=&\ \frac{x^2 + 4x}{x}			&&\underset{\rule{0.8\linewidth}{0.4pt}}{\text{}}\\[2em]
&&	=&\ x + 4					&&\underset{\rule{0.8\linewidth}{0.4pt}}{\text{}}\\[2em]
\end{align*}

% Sørg for at figur er placeret før et vidst sted
\FloatBarrier




