\documentclass[10pt,a4paper]{article}
	
	\usepackage{amsmath,amssymb,amsthm,mathtools,amsfonts}
	\usepackage{breqn}
	\usepackage[utf8]{inputenc}
	\usepackage[danish]{babel}
	\usepackage[T1]{fontenc}
	
	%%%%%%%%%%%%%%%%%%%%%%%%%%%%%%%%%%%%%%%%%%%%%%%%%%%%%%%%%%%%%%%%%%%%%%%%%%%%%% Fonts
	
	\usepackage{fontspec}
	\setmainfont{Times New Roman}
	\newfontfamily\garamond[Numbers=OldStyle]{EB Garamond}
	\newfontfamily\plantin[Numbers=OldStyle]{Plantin MT Pro}
	\newfontfamily\wal{Walbaum Com}
	\newfontfamily\klav{Klavika}
	\newfontfamily\klavl{Klavika light}
	\newfontfamily\quotefont[Ligatures=TeX]{Linux Libertine O}
	
	%%%%%%%%%%%%%%%%%%%%%%%%%%%%%%%%%%%%%%%%%%%%%%%%%%%%%%%%%%%%%%%%%%%%%%%%%%%%%% Colors
	
	\usepackage{xcolor}
	
	\definecolor{hgblue}{RGB}{10, 40, 80}
	\definecolor{hgred}{RGB}{140, 10, 10}
	\definecolor{hggrey}{RGB}{215, 215, 210}
	\definecolor{hggreen}{RGB}{145, 205, 205}
	
	%%%%%%%%%%%%%%%%%%%%%%%%%%%%%%%%%%%%%%%%%%%%%%%%%%%%%%%%%%%%%%%%%%%%%%%%%%%%%% Other Graphics
	
	\usepackage{graphicx}
	\graphicspath{{../img/}}
	\usepackage{tikz}
	\usepackage{placeins}
	
	\usepackage[pdfborder={0 0 0}]{hyperref} %% Ingen firkant rundt om referencer

	\usepackage{multicol}
	\usepackage{cellspace}
		\setlength\cellspacetoplimit{100pt}
		\setlength\cellspacebottomlimit{100pt}
	\usepackage{tabularx}
		\newcolumntype{b}{>{\hsize=\hsize}>{\centering\arraybackslash}X}
	\usepackage{siunitx}
	\usepackage{cellspace}
		\setlength\cellspacetoplimit{4pt}
		\setlength\cellspacebottomlimit{4pt}
	
	\usepackage{enumitem}% better controls with enumerating
	\usepackage{wrapfig}
	\usepackage{caption}
	%\usepackage{dirtytalk}
	%\usepackage{tasks}
	\usepackage{titlesec}

	\setlength{\parskip}{0.5\baselineskip}
	\setlength{\parindent}{0cm}
	\setlist[enumerate]{}
	%\setlist[enumerate]{leftmargin=\parindent}
	%\setlist[enumerate]{nosep}
	
	
	%%%%%%%%%%%%%%%%%%%%%%%%%%%%%%%%%%%%%%%%%%%%%%%%%%%%%%%%%%%%%%%%%%%%%%%%%%%%%% New commands
	
	\newcommand\svar[1]{\newline\textcolor{red}{\textbf{Svar}\\#1}}
	
	\usepackage[h]{esvect}
	\newcommand{\twvect}[2]{\ensuremath{\begin{pmatrix}#1\\#2\end{pmatrix}}}
	
	\newcommand*\quotesize{20}
	\newcommand*{\openquote}
		{\tikz[remember picture,overlay,xshift=-3ex,yshift=-0ex]
			\node (OQ) {\quotefont\fontsize{\quotesize}{\quotesize}\selectfont``};\kern0pt}

	\newcommand*{\closequote}[1]
		{\tikz[remember picture,overlay,xshift=2ex,yshift={#1}]
			\node (CQ) {\quotefont\fontsize{\quotesize}{\quotesize}\selectfont''};}
			
	\newcommand{\qm}[1]{``#1''}
	\newcommand{\iquote}[1]{\begin{quote}\itshape \openquote#1\hfill\closequote{0ex} \end{quote}}
	
	\usepackage{fontawesome}
	\newcommand{\hint}[3][]{%
  		\marginpar{%
   			 \vspace{-#3pt}%	
			 \faBook \ \klavl\scriptsize#2%
			 \ifx\relax#1\relax\else\\[5pt]\faYoutube \ \klavl\scriptsize\color{hgblue}\href{#1}{Videobevis}\fi%
			 	}%
			}

	
	%%%%%%%%%%%%%%%%%%%%%%%%%%%%%%%%%%%%%%%%%%%%%%%%%%%%%%%%%%%%%%%%%%%%%%%%%%%%%% Sections
	
	%\setcounter{secnumdepth}{0}
	\usepackage{remreset} % continued subsection numbering
		\makeatletter
			\@removefromreset{subsection}{section}\renewcommand\thesubsection{\arabic{subsection}}
		\makeatother
	\renewcommand\thesection{\klavl\huge Del \arabic{section}}
	\titleformat{\section}[display]{\centering\klavl}{\thesection}{-10pt}{\scshape\footnotesize}[\vspace{20pt}]
	\renewcommand\thesubsection{}
	\titleformat{\subsection}{\color{hgred}\large\klavl}{}{}{}
	
	%%%%%%%%%%%%%%%%%%%%%%%%%%%%%%s%%%%%%%%%%%%%%%%%%%%%%%%%%%%%%%%%%%%%%%%%%%%%%%% Header & Footer
	
	\usepackage{bigfoot}
	\DeclareNewFootnote{a}
		\renewcommand{\thefootnotea}{\fnsymbol{footnotea}}
			\newcommand{\footnotes}[1]{\footnotea[2]{#1}} % here you can specify what symbol you want in footnotes
			\MakePerPage{footnotes}
	
		\usepackage{fancyhdr}
	\fancypagestyle{plain}{%
		\setlength{\headheight}{60pt}%
		\setlength{\headsep}{0pt}
		\fancyhf{}% No header/footer
		\renewcommand{\footrulewidth}{0pt}% No footer rule
		\renewcommand\headrule{
			\centerline{\begin{minipage}{1\textwidth}
				\color{hgred}\hrule width \hsize \kern 1mm \hrule width \hsize height 2pt 
			\end{minipage}}
		}
		\fancyhead[L]{\klav\color{hgblue} HELSINGØR\\\klavl GYMNASIUM}
		\fancyhead[R]{\klavl\color{hgblue}\footnotesize Sommer 2024}
		\fancyhead[C]{\klav\color{hgblue}\normalsize \klavl 2s Ma: mundtlig eksamen\\\Large Individuelle spørgsmål}
		\cfoot{\thepage}
	}
	\pagestyle{fancy}
	\fancyhf{}
	\renewcommand\headrule{
			\centerline{\begin{minipage}{\textwidth}
				\color{hgred}\hrule width \hsize \kern 1mm
			\end{minipage}}}
	\lhead{\klavl\color{hgblue}\footnotesize 2g Ma}
	\rhead{\klavl\color{hgblue}\footnotesize Sommer 2024}
	\cfoot{\thepage}
	
	%%%%%%%%%%%%%%%%%%%%%%%%%%%%%%%%%%%%%%%%%%%%%%%%%%%%%%%%%%%%%%%%%%%%%%%%%%%%%% Document

\begin{document}

\thispagestyle{plain}
\begin{tikzpicture}[remember picture,overlay]
    \node[anchor=north west,yshift=-39pt,xshift=-132pt]%
        at (current page.north west)
        {\includegraphics[scale=0.3]{HG_logo.PNG}};
    \end{tikzpicture}

Her følger de spørgsmål I vil kunne trække til den mundtlige eksamen i matematik. I vil således kunne forberede en præsentation af en besvarelse af hvert spørgsmål inden selve prøven. Herefter er prøveformen sådan at I trækker ét af spørgsmålene, og skal, efter 24 minutters forberedelse, eksamineres i dette, ligeledes i 24 minutter. God fornøjelse med læsningen.

\clearpage
\subsection*{Vækst}
\noindent
\begin{enumerate}
\item Gør, for en lineær funktion $f(x)=ax+b$, rede for formlerne til beregning af $a$ og $b$, når grafen går gennem punkterne $(x_1,y_1)$ og $(x_2,y_2)$:
\begin{align*}
a&=\frac{y_2-y_1}{x_2-x_1}\\
b&=y_1-a\cdot x_1
\end{align*}
Giv eksempler på lineære funktioner og forklar hvad $a$ og $b$ betyder i de konkrete eksempler. \hint[https://youtu.be/dzidekuK7DM]{G: Sætning 5\\Eksempel 127}{95}
\end{enumerate}
\clearpage
\subsection*{Vækst}

\begin{enumerate}[resume]
\item Gør, for en eksponentiel funktion $f(x)=b\cdot a^x$, rede for formlerne til beregning af $a$ og $b$ når grafen går gennem punkterne $(x_1, y_1)$ og $(x_2, y_2)$:
\begin{align*}
a&=\sqrt[x_2-x_1]{\frac{y_2}{y_1}}\\
b&=\frac{y_1}{a^{x_1}}
\end{align*}
Giv eksempler på eksponentielle funktioner og forklar hvad $a$ og $b$ betyder i de konkrete eksempler. \hint[https://youtu.be/5Rm0jLpzack]{B1: Sætning 1.3\\Eksempel 130}{105}
\end{enumerate}
\clearpage
\subsection*{Vækst}

\begin{enumerate}[resume]
\item Gør, for en eksponentiel funktion $f(x)=b\cdot a^x$, rede for formlen til beregning af fordoblingskonstanten:
$$T_2=\frac{\log{2}}{\log{a}}$$
Giv eksempler på eksponentielle funktioner og forklar hvad fordoblings- og halveringskonstanten betyder i de konkrete eksempler.\hint[https://youtu.be/e1k5ZrpcYr4]{B1: Sætning 1.5\\Eksempel 142}{65}
\end{enumerate}
\clearpage
\subsection*{Vækst}

\begin{enumerate}[resume]
\item Gør, for en potens funktion $f(x)=b\cdot x^a$, rede for formlerne til beregning af $a$ og $b$ når grafen går gennem punkterne $(x_1,y_1)$ og $(x_2,y_2)$:
\begin{align*}
a&=\frac{\log{\frac{y_2}{y_1}}}{\log{\frac{x_2}{x_1}}}\\
b&=\frac{y_1}{x_1^a}
\end{align*}
Giv eksempler på potenssammenhænge fra videnskaben. \hint[https://youtu.be/opwOct5Oufw]{B1: Sætning 1.7\\Eksempel 156}{90}
\end{enumerate}
\clearpage
\subsection*{Vækst}

\begin{enumerate}[resume]
\item Definer lineær vækst, eksponentiel vækst og potensvækst. Bevis for hver af de tre væksttyper hvad der sker med $y$-værdien når: \hint{G: Sætning 3\\B1: Sætning 1.4\\B1: Sætning 1.8}{20}
\begin{description}%[\alpha)]
\item[Lineær vækst:] $x$-værdien øges med 1 \quad (Udregn $f(x+1)$)
\item[Eksponentiel vækst:] $x$-værdien øges med 1 \quad (Udregn $f(x+1)$)
\item[Potensvækst:] $x$-værdien ganges med $k$ \quad (Udregn $f(x\cdot k)$)
\end{description}
\end{enumerate}

\subsection*{Polynomier}
\begin{enumerate}[resume]%\vspace{-25pt}
\item Gør, for et andengradspolynomium $f(x)=ax^2+bx+c$, rede for formlerne til udregning af polynomiets rødder:
$$x=\frac{-b\pm\sqrt{b^2-4ac}}{2a},$$
\hint[https://youtu.be/ZuI1ZyKr0Yg]{B1: Sætning 2.2}{65}
og forklar hvordan grafen for andengradspolynomiet afhænger af $a$ og diskriminanten $d$. 
\item Gør, for et andengradspolynomium $f(x)=ax^2+bx+c$, rede for formlerne til udregning af polynomiets toppunkt:
$$T=\left(\frac{-b}{2a},\frac{-d}{4a}\right),$$
og forklar, hvordan grafen for andengradspolynomiet afhænger af $a$, $b$ og $c$.\hint[https://youtu.be/R10bPzbuP2Y?si=Lcs7802DJc1o4Lrw]{B1: Sætning 2.5}{65}
\end{enumerate}


\subsection*{Geometri}
\begin{enumerate}[resume]%\vspace{-25pt}
\item Gør, for en ret linje i et koordinatsystem der går gennem punktet $(x_0,y_0)$ og har $\vec{n}=\twvect{a}{b}$ som normalvektor, rede for ligningen:
$$a(x-x_0)+b(y-y_0)=0$$
og forklar, med et eksempel, sammenhængen mellem denne og forskriften for en lineær funktion. \hint{B2: Sætning 4.2}{80}

\item Gør, for en ret linje i et koordinatsystem der går gennem punktet $(x_0,y_0)$ og har $\vec{r}=\twvect{r_1}{r_2}$ som retningsvektor, rede for denne linjes parameterfremstilling:
$$\twvect{x}{y}=\twvect{x_0}{y_0}+t\cdot\twvect{r_1}{r_2}$$
og forklar, med et eksempel, hvordan man kan finde et skæringspunkt mellem to linjer.\hint{B2: Sætning 4.1\\Eksempel 413}{90}

\item Gør, for en cirkel i et koordinatsystem med centrum i $(x_0,y_0)$ og radius $r$, rede for cirklens ligning
$$(x-x_0)^2+(y-y_0)^2=r^2$$
og forklar, med et eksempel, hvordan man kan afgøre om en linje er tangent til en cirkel. \hint{B2: Sætning 4.7\\Eksempel 438}{50}

\item Gør rede for definitionen af cosinus og sinus, og udled formlerne til beregning af vinkler og sider i retvinklede trekanter:
\begin{align*}
\cos(v)&=\frac{\text{hosliggende katete}}{\text{hypotenuse}}\\
\sin(v)&=\frac{\text{modstående katete}}{\text{hypotenuse}}
\end{align*}
\hint[https://youtu.be/3dqk5zOMA_Y]{B1: Sætning 4.14\\Eksempel 432\\Eksempel 435}{85}
\item Gør rede for formlerne til beregning af areal i vilkårlige trekanter:
\begin{align*}
T&=\frac{1}{2} \cdot a\cdot b\cdot \sin(C)\\
T&=\frac{1}{2} \cdot b\cdot c\cdot \sin(A)\\
T&=\frac{1}{2} \cdot c\cdot a\cdot \sin(B)
\end{align*}
og udled på baggrund heraf sinusrelationerne:
$$\frac{\sin{A}}{a}=\frac{\sin{B}}{b}=\frac{\sin{C}}{c}$$
\hint{B1: Sætning 4.17 - 4.19\\Eksempel 451}{125}
\end{enumerate}

\subsection*{Differentialregning}
\begin{enumerate}[resume]
\item Gør, for funktionen $f(x)=x^2$, rede for formlen for dennes afledte i $x_0$:
$$f'(x_0)=2\cdot x_0$$
og forklar hvordan differentialregning kan bruges til at løse et geometrisk optimeringsproblem. \hint[https://youtu.be/ENGoqNTmCOQ]{B2: Sætning 5.1}{55}

\item Gør, for funktionen $f(x)=g(x)+h(x)$, rede for formlen for dennes afledte i $x_0$:
$$f'(x_0)=g'(x_0)+h'(x_0)$$
og forklar hvordan differentialregning kan bruges til at løse et geometrisk optimeringsproblem. \hint{B2: Sætning 5.7}{58}
\end{enumerate}

\subsection*{Sandsynlighedsregning}
\begin{enumerate}[resume]
\item Gør, for en binomialfordelt stokastisk variabel $X$, rede for formlen til udregning af sandsynligheden:
$$P(X=r)=K(n,r)\cdot p^r\cdot (1-p)^{n-r}$$
og forklar hvordan man med en såkaldt binomialtest kan afprøve en statistisk hypotese. \hint{B2: Sætning 3.6\\Eksempel 433}{75}

\item Gør, for en binomialfordelt stokastisk variabel $X$, rede for formlen til udregning af sandsynligheden:
$$P(X=r)=K(n,r)\cdot p^r\cdot (1-p)^{n-r}$$
og forklar, med et eksempel, hvordan man med et såkaldt konfidensinterval kan undersøge parametre i hele populationer ud fra stikprøver. \hint{B2: Sætning 3.6\\Eksempel 332}{63}
\end{enumerate}












\end{document}

%%%%%%%%%%%%%%%%%%%%%%%%%%%%%%%%%%%%%%%%%%%%%%%%%%%%%%%%%%%%%%%%%%%%%%%%%%%%%% Skabeloner

% Sidestillet figur
\begin{wrapfigure}[8]{r}{0.2\textwidth}
\vspace{-18pt}
\includegraphics[width=0.3\textwidth]{ovn}
\end{wrapfigure}

% Midterstillet figur
\begin{figure}[h!]
\centering
\includegraphics[width=0.8\textwidth]{abc}
\end{figure}

% Førstillet figur
\
\begin{figure}[h!]
\centering
\vspace{-15pt}
\includegraphics[width=0.35\textwidth]{stud}
\end{figure}

% Tabel
\begin{figure}[h!]
\centering\renewcommand{\arraystretch}{1.5}
\begin{tabularx}{0.9\textwidth}{|l|b|b|b|b|b|b|}
	\hline \cellcolor{hggreen} Decimaltal & 7\% & -51\% & 13,7\% & 126\% & 456\% & 0,28\%\\\hline
	\cellcolor{hggreen} Procenttal &  &  &  &  &  & \\\hline
\end{tabularx}
\end{figure}

% Sørg for at figur er placeret før et vidst sted
\FloatBarrier




